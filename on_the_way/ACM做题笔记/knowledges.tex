%knowledges
\section{知识点总结}
\subsection{复杂度}
复杂度比较常见的函数复杂度比较:
\begin{equation}
c < \log n < n < n\log n < n^d < a^n < n!	\nonumber
\end{equation}
递归算法的时间复杂的情况有点复杂,其主定理由该公式计算:
\begin{equation}
T(n) = aT(\frac{n}{b}) + f(n)  	\nonumber
\end{equation}
a,b都是常数,n是规模变量,一般$f(n)=\mathcal{O}(n^d)$,d也是常数。
经过推理\ref{recursion},可得一下结论:
\begin{itemize}
\item 若$d < \log_b^a$,$T(n)=\mathcal{O}(n^d)$
\item 若$d = \log_b^a$,$T(n)=\mathcal{O}(n^d\log n)$
\item 若$d > \log_b^a$,$T(n)=\mathcal{O}(n^{\log_b^a})$
\end{itemize}
附粗糙推理过程:
假定$b^t=n$,即$t=\log_b^n$。
\begin{eqnarray}
\label{recursion}
T(n) &=& aT(\frac{n}{b}) + f(n)  	\\
&=& a^2T(\frac{n}{b^2}) + af(\frac{n}{b}) + f(n) \\
&=& a^tT(1) + \sum_{i=0}^{t-1} a^i b^{(t-i)*d} \\
&=& a^tT(1) + \frac{b^d}{b^d-a}(b^{dt}-a^t) \\
&=& [T(1)-\frac{b^d}{b^d-a}] n^{\log_b^a} + \frac{b^d}{b^d-a}n^d
\end{eqnarray}
\begin{itemize}
\item 若$d < \log_b^a$,则等式5$=\mathcal{O}(n^d)$。
\item 若$d = \log_b^a$,则等式3$=a^t(t+T(1))=n^{\log_b^a}(\log_b^n+T(1))=\mathcal{O}(n^d\log n)$。
\item 若$d > \log_b^a$,则等式5$=\mathcal{O}(n^{\log_b^a})$。
\end{itemize}




