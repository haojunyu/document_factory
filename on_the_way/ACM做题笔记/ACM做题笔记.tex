\documentclass[10pt,a4paper]{article}	
\usepackage{amsmath}
\usepackage{amsfonts}
\usepackage{amssymb}
\usepackage{CJK}													% 支持中文
\usepackage{listings}											% 支持代码显示
\usepackage{color}												% 给文字,表格和图形上色(68种)
\usepackage{xcolor}												% color包的扩展
\usepackage[colorlinks,citecolor = blue, linkcolor=blue,hyperindex,CJKbookmarks]{hyperref}	% 链接功能

%===========================c语言代码显示设置===============================
\lstdefinestyle{C}{
language=[ANSI]c,							% c语言
basicstyle=\small,							% 小号字体
numbers=left,								% 代码左边显行号
keywordstyle=\color{blue},					% 关键字用蓝色显示
numberstyle={\tiny\color{green}},			% 行号小小号,绿色	
numbersep=5pt,								% 行号与代码的距离
commentstyle=\small\color{red},				% 注释颜色和字号
backgroundcolor=\color[rgb]{0.95,1.0,1.0},	% 设置背景颜色
showspaces=false,							% 不显示空格
showtabs=false,								% 不显示\t
tabsize=4,									% \t的长短
frame=shadowbox, 							% 添加外框
framexleftmargin=5mm, 						% 外框左边的页边空白
rulesepcolor=\color{red!20!green!20!blue!20!},	%设置阴影颜色
breaklines=true,								% 自动断行
morekeywords={MYSQL,MYSQL_RES,MYSQL_ROW},
escapeinside=``,
extendedchars=false 							% 解决代码跨页时,章节标题,页眉等汉字不显示的问题
}
%========================SHELL代码显示设置=============================
\lstdefinestyle{BASH}{
language=bash,								% bash代码
basicstyle=\small,							% 小号字体
numbers=left,								% 代码左边显行号
keywordstyle=\color{blue},					% 关键字用蓝色显示
numberstyle={\tiny\color{green}},			% 行号小小号,绿色	
numbersep=5pt,								% 行号与代码的距离
commentstyle=\small\color{red},				% 注释颜色和字号
backgroundcolor=\color[rgb]{1.0,0.95,1.0},	% 设置背景颜色
showspaces=false,							% 不显示空格
showtabs=false,								% 不显示\t
tabsize=4,									% \t的长短
frame=shadowbox, 							% 添加外框
framexleftmargin=5mm, 						% 外框左边的页边空白
rulesepcolor=\color{red!20!green!20!blue!20!},	%设置阴影颜色
breaklines=true,								% 自动断行
escapeinside=``,
extendedchars=false, 						% 解决代码跨页时,章节标题,页眉等汉字不显示的问题
morekeywords={cp,xz,tar},
literate={\$}{{\textcolor{red}{\$}}}1
         {:}{{\textcolor{gray}{:}}}1
         {hjy@jy}{{\textcolor{orange}{hjy@jy}}}6
         {sudo}{{\textcolor{red}{sudo}}}4,
}
%========================================================================

\begin{document}

\begin{CJK*}{UTF8}{gkai}
%============================++题目和作者++================================
\title{ACM做题笔记}							   					% 题目
\author{郝俊禹\thanks{Email:haojunyu2012@gmail.com}}				% 作者
%============================++++++++++++=================================
\date{}                                             				% 显示作者,不显示日期
\maketitle                                          				% 生成标题
\tableofcontents 												% 生成目录
\clearpage

%knowledges
\section{知识点总结}
\subsection{复杂度}
复杂度比较常见的函数复杂度比较:
\begin{equation}
c < \log n < n < n\log n < n^d < a^n < n!	\nonumber
\end{equation}
递归算法的时间复杂的情况有点复杂,其主定理由该公式计算:
\begin{equation}
T(n) = aT(\frac{n}{b}) + f(n)  	\nonumber
\end{equation}
a,b都是常数,n是规模变量,一般$f(n)=\mathcal{O}(n^d)$,d也是常数。
经过推理\ref{recursion},可得一下结论:
\begin{itemize}
\item 若$d < \log_b^a$,$T(n)=\mathcal{O}(n^d)$
\item 若$d = \log_b^a$,$T(n)=\mathcal{O}(n^d\log n)$
\item 若$d > \log_b^a$,$T(n)=\mathcal{O}(n^{\log_b^a})$
\end{itemize}
附粗糙推理过程:
假定$b^t=n$,即$t=\log_b^n$。
\begin{eqnarray}
\label{recursion}
T(n) &=& aT(\frac{n}{b}) + f(n)  	\\
&=& a^2T(\frac{n}{b^2}) + af(\frac{n}{b}) + f(n) \\
&=& a^tT(1) + \sum_{i=0}^{t-1} a^i b^{(t-i)*d} \\
&=& a^tT(1) + \frac{b^d}{b^d-a}(b^{dt}-a^t) \\
&=& [T(1)-\frac{b^d}{b^d-a}] n^{\log_b^a} + \frac{b^d}{b^d-a}n^d
\end{eqnarray}
\begin{itemize}
\item 若$d < \log_b^a$,则等式5$=\mathcal{O}(n^d)$。
\item 若$d = \log_b^a$,则等式3$=a^t(t+T(1))=n^{\log_b^a}(\log_b^n+T(1))=\mathcal{O}(n^d\log n)$。
\item 若$d > \log_b^a$,则等式5$=\mathcal{O}(n^{\log_b^a})$。
\end{itemize}





%%summary
\section{做题总结}
\subsection{输入输出}
我们平时编程输入输出都是在一个黑框框里面完成的,这样带来的麻烦是每次调试时都得手动的输入测试数据,导致效率很低。
而对于ACM这样追求效率的比赛而言是不能容忍的.这里推荐采用重新定义输入流为文本文件输入的方式,一方面不必在黑框框
中手动输入,另一方面只要注释掉重定义的这行代码,就可以直接提交,非常方便.
\begin{lstlisting}[style=C]
#include<stdio.h>

int main(){
	freopen("in.txt","r",stdin);			//redirect the input stream
	//freopen("out.txt","w",stdout);		//redirect the output stream
	
	int a,b;
	while(scanf("%d%d",&a,&b)!=EOF){
		printf("%d\n",a+b);
	}
	return 0;
}
\end{lstlisting}
\underline{\color{purple}注意点}
\begin{enumerate}
\item 输入输出要用$scanf$和$printf$,少用$cin$和$cout$.前者效率远高于后者.
\item in.txt要放在和执行文件(不是.cpp文件)相同的目录.
\end{enumerate}

\subsection{测评机的状态}
在平台上提交代码后,测评系统会显示程序的文件大小,运行时间,所需内存,以及\textbf{程序的运行结果}等,下面对运行结果
作一下说明.\\
\begin{table}[htbp]
	\caption{\label{tab:STATUS}评测机状态}
    	\begin{center}
        \begin{tabular*}{0.95\textwidth}{@{\extracolsep{\fill}} c c c}
        \hline
        Verdict		&	Abbreviation		&		Indication      	\\
        \hline
        Accepted		&		 AC			&			通过			\\
        Presentation Error &  PE			&		答案正确,格式错误	\\
		Time Limit Exceeded& TLE         &			超时			\\
		Memory Limit Exceeded&	MLE		&		内存超过限制		\\
		Wrong Answer	&		 WA			&		答案不对			\\
		Runtime Error	&	 RE			&		一般是由于越界访问导致的\\
		Output Limit Exceeded & OLE		&		输出文件超限,一般是因为死循环\\
		Compile Error	&	 CE			&		编译出错			\\
		System Error		&	 SE			&		尚未遇到过		\\
		Validator Error	& 	 VE			&		好像是你意图不轨的意思\\	
        \hline
        \end{tabular*}
    \end{center}
\end{table}


\subsection{编译及调试}
作为C/C++的集成环境有很多,比如VC++6.0,VS系列等,不过推荐使用codeblocks,因为它是跨平台的,
wins和linux下都有相应的安装程序,而且从使用方便程度上讲丝毫不弱VC++6.0,VS等.
至于使用详见本主机$\$HOME/Public/resources/codeblocks$目录下的三个文件.
\begin{description}
\item[基本篇]		CodeBlock\_{}ACM.pdf(基本的调试)
\item[进阶篇] 	CodeBlock中文版使用手册.pdf(常用的操作)
\item[大成篇]		manual\_{}en\_{}codeblocks.chm(英文帮助文档) 
\end{description}


\subsection{C++头文件及常用函数}
\subsubsection{iostream}
数据的输入输出流,常用的函数有:
\begin{description}
\item[cin]	输入
\item[cout]	输出
\end{description}


\subsubsection{cstdio}
C的输入输出,等价于C中stdio.h,常用的函数有:
\begin{description}
\item[scanf]		输入(效率比cin高很多)	
\begin{lstlisting}[style=C]
#include<stdio.h>

int main(){
	int a;
	scanf("%d",&a);
	printf("%d",a);	
	return 0;
}
\end{lstlisting}
\item[printf]	输出(效率比cout高很多)
\item[freopen]	文件输入输出的重定向
\end{description}


\subsubsection{cstring}
字符串操作函数,等价于C中string.h,常用的函数有:
\begin{description}
\item[strcpy]	字符串拷贝
\item[strcmp]	字符串比较
\item[strcat]	字符串连接
\item[memset]	内存初始化
\end{description}


\subsubsection{string}
C++中的string类,不能用strcpy等c函数去操作.常用函数有:
\begin{description}
\item[func]		none
\end{description}


\subsubsection{bitset}
标准模板库里的位操作类
\begin{description}
\item[func]		none
\end{description}


\subsubsection{STL-vector}
标准模板库里的向量类
\begin{description}
\item[func]		none
\end{description}


\subsubsection{STL-stack}
标准模板库里的堆栈类
\begin{description}
\item[func]		none
\end{description}


\subsubsection{STL-queue}
标准模板库里的队列类
\begin{description}
\item[func]		none
\end{description}


\subsubsection{STL-list}
标准模板库里的链表类
\begin{description}
\item[func]		none
\end{description}


\subsubsection{STL-map}
标准模板库里的哈希表类
\begin{description}
\item[func]		none
\end{description}


\subsubsection{numeric}
常用数字操作,一般和algorithm搭配使用.
\begin{description}
\item[func]		none
\end{description}


\subsubsection{STL-algorithm}
标准模板库里的各类算法类
\begin{description}
\item[swap]		交换
\item[sort]		排序
\item[merge]		归并
\item[max]		最大值
\item[min]		最小值
\item[binary\_{}search]	二分查找
\end{description}


\subsubsection{STL-functional}
标准模板库里的定义运算的函数(代替运算符)
\begin{description}
\item[func]		none
\end{description}


\subsubsection{STL-map}
标准模板库里的哈希表类
\begin{description}
\item[func]		none
\end{description}

\subsection{C\&{}C++的区别}
\subsubsection{头文件}
C语言中统一使用$filename.h$格式的头文件,而C++中使用$filename$格式的头文件,如C中
的$\#include<iostream.h>$,C++中的$\#include<iostream>$.不过C++
中新定义的方法都是有命名空间的$using namespace std;$.而且C++中兼容C程序,所以C中
的$\#include<math.h>$在C++中也有对应的重新定义的方法在$\#include<cmath>$中.

\subsubsection{数据类型}
C语言中没有bool类型,所以要自己重新定义,代码如下:
\begin{lstlisting}[style=C]
#define bool int
#define false 0
#define true 1
\end{lstlisting}
这样就可以C++中统一起来.

\subsubsection{编译}
C语言中对语法要求比较高,它用gcc命令进行编译,在$int main$的主函数中要在最后添加$return 0;$语句.
而C++用g++命令进行编译,并不会报出要添加$return 0;$的警告.



















\clearpage
%%hdu
\section{杭电ACM}
\subsection{Problem Archive}
\subsubsection{1041-Computer Transformation\footnote{\url{http://acm.hdu.edu.cn/showproblem.php?pid=1041}} }
\begin{itemize}
\item Description \\
A sequence consisting of one digit, the number 1 is initially written into a computer. At each successive time step, the computer simultaneously tranforms each digit 0 into the sequence 1 0 and each digit 1 into the sequence 0 1. So, after the first time step, the sequence 0 1 is obtained; after the second, the sequence 1 0 0 1, after the third, the sequence 0 1 1 0 1 0 0 1 and so on. 

How many pairs of consequitive zeroes will appear in the sequence after n steps? 
\item Input	\\
Every input line contains one natural number n ($0 < n \leq1 000$).\\
sample input:
\begin{lstlisting}[style=C]
2
3
\end{lstlisting}
\item Output	\\
For each input n print the number of consecutive zeroes pairs that will appear in the sequence after n steps.\\
sample output:
\begin{lstlisting}[style=C]
1
1
\end{lstlisting}
\item Analyze	\\
经过分析易知:
\begin{enumerate}
\item 第n步中1的个数为$g(n)=2^{n-1}$
\item $consequitive zeroes$只能是两个0的组合,因为0或1都无法产生00,所以$consequitive zeroes$的个数就是0 0的个数
\item 产生0 0的可能为1\rightarrow{}01\rightarrow{}1001;00\rightarrow{}1010\rightarrow{}01100110

\end{enumerate}
假设用$f(n)$代表第n步中连续0的个数,那么易知等式$f(n)=f(n-2)+g(n-2)$.

\item Code	\\
\begin{lstlisting}[style=C]

\end{lstlisting}ng}[styl
\item Result		\\
\end{itemize}

\clearpage














\subsubsection{xxxx-template\footnote{\url{http://acm.hdu.edu.cn/showproblem.php?pid=1041}} }
\begin{itemize}
\item Description \\
A sequence consisting of one digit, the number 1 is initially written into a computer. At each successive time step, the computer simultaneously tranforms each digit 0 into the sequence 1 0 and each digit 1 into the sequence 0 1. So, after the first time step, the sequence 0 1 is obtained; after the second, the sequence 1 0 0 1, after the third, the sequence 0 1 1 0 1 0 0 1 and so on. 

How many pairs of consequitive zeroes will appear in the sequence after n steps? 
\item Input	\\
Every input line contains one natural number n ($0 < n \leq1 000$).\\
sample input:
\begin{lstlisting}[style=C]
2
3
\end{lstlisting}
\item Output	\\
For each input n print the number of consecutive zeroes pairs that will appear in the sequence after n steps.\\
sample output:
\begin{lstlisting}[style=C]
1
1
\end{lstlisting}
\item Analyze	\\
\item Code	\\
\item Result		\\
\end{itemize}	\\



\clearpage
%%ecnu
\section{华师大ACM}


\clearpage     
\end{CJK*}
\end{document}