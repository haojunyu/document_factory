%hdu
\section{杭电ACM}
\subsection{Problem Archive}
\subsubsection{1041-Computer Transformation\footnote{\url{http://acm.hdu.edu.cn/showproblem.php?pid=1041}} }
\begin{itemize}
\item Description \\
A sequence consisting of one digit, the number 1 is initially written into a computer. At each successive time step, the computer simultaneously tranforms each digit 0 into the sequence 1 0 and each digit 1 into the sequence 0 1. So, after the first time step, the sequence 0 1 is obtained; after the second, the sequence 1 0 0 1, after the third, the sequence 0 1 1 0 1 0 0 1 and so on. 

How many pairs of consequitive zeroes will appear in the sequence after n steps? 
\item Input	\\
Every input line contains one natural number n ($0 < n \leq1 000$).\\
sample input:
\begin{lstlisting}[style=C]
2
3
\end{lstlisting}
\item Output	\\
For each input n print the number of consecutive zeroes pairs that will appear in the sequence after n steps.\\
sample output:
\begin{lstlisting}[style=C]
1
1
\end{lstlisting}
\item Analyze	\\
经过分析易知:
\begin{enumerate}
\item 第n步中1的个数为$g(n)=2^{n-1}$
\item $consequitive zeroes$只能是两个0的组合,因为0或1都无法产生00,所以$consequitive zeroes$的个数就是0 0的个数
\item 产生0 0的可能为1\rightarrow{}01\rightarrow{}1001;00\rightarrow{}1010\rightarrow{}01100110

\end{enumerate}
假设用$f(n)$代表第n步中连续0的个数,那么易知等式$f(n)=f(n-2)+g(n-2)$.

\item Code	\\
\begin{lstlisting}[style=C]

\end{lstlisting}ng}[styl
\item Result		\\
\end{itemize}

\clearpage














\subsubsection{xxxx-template\footnote{\url{http://acm.hdu.edu.cn/showproblem.php?pid=1041}} }
\begin{itemize}
\item Description \\
A sequence consisting of one digit, the number 1 is initially written into a computer. At each successive time step, the computer simultaneously tranforms each digit 0 into the sequence 1 0 and each digit 1 into the sequence 0 1. So, after the first time step, the sequence 0 1 is obtained; after the second, the sequence 1 0 0 1, after the third, the sequence 0 1 1 0 1 0 0 1 and so on. 

How many pairs of consequitive zeroes will appear in the sequence after n steps? 
\item Input	\\
Every input line contains one natural number n ($0 < n \leq1 000$).\\
sample input:
\begin{lstlisting}[style=C]
2
3
\end{lstlisting}
\item Output	\\
For each input n print the number of consecutive zeroes pairs that will appear in the sequence after n steps.\\
sample output:
\begin{lstlisting}[style=C]
1
1
\end{lstlisting}
\item Analyze	\\
\item Code	\\
\item Result		\\
\end{itemize}	\\



\clearpage