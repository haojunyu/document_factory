%socket
\section{socket网络编程}
\subsection{常见结构体}
\subsubsection{sockaddr}
%\lstinputlisting[lastline=4]{main.c}
\begin{lstlisting}[style=C]
struct sockaddr {
	unsigned  short  sa_family;     	//address family="AF_XXX",often use AF_INET
	char  sa_data[14];  				//14 characters of address.            
};
\end{lstlisting}
sockaddr该结构体用作bind、connect、recvfrom、sendto等函数的参数,指明地址信息。但一般编程中不直接对此数据结构操作,而是使用与sockaddr等价的数据结构sockaddr\_{}in(netinet/in.h)

\subsubsection{sockaddr\_{}in}
\begin{lstlisting}[style=C]
struct  sockaddr_in {
	short  int  sin_family; 				/* Address family */
	unsigned  short  int  sin_port; 		/* Port number */
	struct  in_addr  sin_addr;			/* Internet address [htons(INADDR_ANY)]*/
	unsigned  char  sin_zero[8]; 		/* Same size as struct sockaddr(16 byte) */
};
\end{lstlisting}
\begin{lstlisting}[style=C]
struct  in_addr {
	unsigned  long  s_addr;				/* address in network byte order */
};
\end{lstlisting}
填值的时候使用sockaddr\_{}in结构,而作为函数的参数传入的时候转换成sockaddr结构就行了,毕竟都是16个字符长。


\subsection{主要函数}
\subsubsection{socket}
\begin{lstlisting}[style=C]
#include<sys/types.h>
#include<sys/socket.h>
int socket(int domain,int type,int protocol);
\end{lstlisting}
\begin{description}
\item[功能]	使用前创建一个新的套接字
\item[参数]		 
\begin{description}
\item[domain] 域名,一般设置为“AF\_{}INET”(ipv4)
\item[type]	套接口类型,一般为SOCK\_{}STREAM(tcp)或SOCK\_{}DGRAM(udp)
\item[protocol]	协议,一般设置为0
\end{description}
\item[返回值]		返回一个套接口描述符,如果出错,则返回­1。
\end{description}

\subsubsection{bind}
\begin{lstlisting}[style=C]
#include<sys/types.h>
#include<sys/socket.h>
int bind(int sockfd,struct sockaddr* my_addr,int addrlen);
\end{lstlisting}
\begin{description}
\item[功能]	把套接口绑定到本地计算机的某一个端口上。但如果你只想使用
connect()则无此必要。
\item[参数]		 
\begin{description}
\item[sockfd] 	由socket()调用返回的套接口文件描述符
\item[my\_{}addr]	指向数据结构sockaddr的指针,sockaddr中包括了关于你的地址、端口和IP地址的信息。
\item[addrlen]	地址长度,一般设置为sizeof(structsockaddr)。
\end{description}
\item[返回值]		没有错误,就返回0,否则返回-1
\end{description}

\subsubsection{connect}
\begin{lstlisting}[style=C]
#include<sys/types.h>
#include<sys/socket.h>
int connect(int sockfd,struct sockaddr* serv_addr,int addrlen);
\end{lstlisting}
\begin{description}
\item[功能]	完成面向连接的协议的连接过程。在建立连接的时候总会有一方先发送数据,
那么谁调用了connect谁就是先发送数据的一方。
\item[参数]		 
\begin{description}
\item[sockfd]	由socket()调用返回的套接口文件描述符
\item[serv\_{}addr]	指向数据结构sockaddr的指针,sockaddr中包括了关于你的地址、端口和IP地址的信息。
\item[addrlen]	地址长度,一般设置为sizeof(structsockaddr)。
\end{description}
\item[返回值]		没有错误,就返回0,否则返回SOCKET\_{}ERROR(-1)
\end{description}


\subsubsection{listen}
\begin{lstlisting}[style=C]
#include<sys/types.h>
#include<sys/socket.h>
int listen(int sockfd,int backlog);
\end{lstlisting}
\begin{description}
\item[功能]	用于面向连接服务器,表明它愿意接收连接.一般先调用listen(),然后再调用accept()。
\item[参数]		 
\begin{description}
\item[sockfd]	由socket()调用返回的套接口文件描述符
\item[backlog]	进入队列中允许的连接的个数。进入的连接请求在使用系统调用
accept()应答之前要在进入队列中等待。这个值是队列中最多可以拥有的请求的
个数。大多数系统的缺省设置为20。你可以设置为5或者10。
\end{description}
\item[返回值]		没有错误,就返回0,否则返回SOCKET\_{}ERROR(-1)
\end{description}


\subsubsection{accept}
\begin{lstlisting}[style=C]
#include<sys/socket.h>
int accept(int sockfd,void* addr,int* addrlen);
\end{lstlisting}
\begin{description}
\item[功能]	客户端用connect()连接,服务端用listen()监听端口。但此连接将会在队列中等待,直到服务端使用accept()处理
它。调用accept()之后,将会返回一个全新的套接口文件描述符来处理这个单个的连接。
\item[参数]		 
\begin{description}
\item[sockfd]	由socket()调用返回的套接口文件描述符
\item[addr]		addr是指向本地的数据结构sockaddr\_{}in的指针,客户端的信息存储在其中
\item[addrlen]	地址长度,一般为sizeof(structsockaddr)。
\end{description}
\item[返回值]		没有错误,就返回一个全新的套接口文件描述符,否则返回-1。
\end{description}


\subsubsection{send[tcp]}
\begin{lstlisting}[style=C]
#include<sys/socket.h>
int send(int sockfd,const void* msg,int len,int flags);
\end{lstlisting}
\begin{description}
\item[功能]	客户端发送数据
\item[参数]		 
\begin{description}
\item[sockfd]	由socket()调用返回的套接口文件描述符,可以是通过socket()
系统调用返回的,也可以是通过accept()系统调用得到的。
\item[msg]		指向你希望发送的数据的指针。
\item[len]		数据的字节长度
\item[flags]		参数标志,一般设置为0。
\end{description}
\item[返回值]		返回实际发送的字节数,这可能比你实际想要发送的字节数少。
如果返回的字节数比要发送的字节数少,你在以后必须发送剩下的数据。当
send()出错时,将返回­1。
\end{description}


\subsubsection{sendto[udp]}
\begin{lstlisting}[style=C]
#include<sys/socket.h>
ssize_t sendto(int sockfd, void *buf, size_t len, int flags,struct sockaddr *src_addr, socklen_t *addrlen);
\end{lstlisting}
\begin{description}
\item[功能]	客户端发送数据(常用于udp连接)
\item[参数]		 
\begin{description}
\item[sockfd]	标识一个已连接套接口的描述字。
\item[buff]		接收数据缓冲区。 
\item[len]		缓冲区长度。 
\item[flags]		参数标志,一般设置为0。
\item[src\_{}addr]	指向装有源地址的缓冲区。
\item[addren]	指向装有源地址的缓冲区。	
\end{description}
\item[返回值]		返回实际发送的字节数,这可能比你实际想要发送的字节数少。
如果返回的字节数比要发送的字节数少,你在以后必须发送剩下的数据。当
sendto()出错时,将返回­1。
\end{description}


\subsubsection{recv[tcp]}
\begin{lstlisting}[style=C]
#include<sys/socket.h>
int recv(int sockfd,void* buf,int len,unsigned int flags);
\end{lstlisting}
\begin{description}
\item[功能]	服务端接受数据(常用于tcp连接)
\item[参数]		 
\begin{description}
\item[sockfd]	由socket()调用返回的套接口文件描述符,可以是通过socket()
系统调用返回的,也可以是通过accept()系统调用得到的。
\item[msg]		指向你希望发送的数据的指针。
\item[len]		缓冲区的最大长度
\item[flags]		参数标志,一般设置为0。
\end{description}
\item[返回值]		返回实际读取到缓冲区的字节数,如果出错则返回­1。
\end{description}


\subsubsection{recvfrom[udp]}
\begin{lstlisting}[style=C]
#include<sys/socket.h>
ssize_t recvfrom(int sockfd, void *buf, size_t len, int flags,struct sockaddr *src_addr, socklen_t *addrlen);
\end{lstlisting}
\begin{description}
\item[功能]	服务端接受数据(常用于udp连接)
\item[参数]		 
\begin{description}
\item[sockfd]	标识一个已连接套接口的描述字。
\item[buff]		接收数据缓冲区。 
\item[len]		缓冲区长度。 
\item[flags]		参数标志,一般设置为0。
\item[src\_{}addr]	指向装有源地址的缓冲区。
\item[addren]	指向装有源地址的缓冲区。	
\end{description}
\item[返回值]		返回实际读取到缓冲区的字节数,如果出错则返回­1。
\end{description}


\subsubsection{close}
\begin{lstlisting}[style=C]
#include<unistd.h>
int close(int sockfd);
\end{lstlisting}
\begin{description}
\item[功能]	关闭本进程的socketfd,但链接还是开着的,用这个socketfd的其它进程还能用这个链接,能读或写这个socketfd。
\item[参数]		 
\begin{description}
\item[sockfd]	由socket()调用返回的套接口文件描述符,可以是通过socket()
系统调用返回的,也可以是通过accept()系统调用得到的。
\end{description}
\item[返回值]		成功则返回0,错误返回-1。
\end{description}


\subsubsection{shutdown}
\begin{lstlisting}[style=C]
#include<sys/socket.h>
int shutdown(int sockfd,int how);
\end{lstlisting}
\begin{description}
\item[功能]	破坏了socket链接,读的时候可能侦探到EOF结束符,写的时候可能会收到一个SIGPIPE信号,这个信号可能直到
socket buffer被填充了才收到,shutdown还有一个关闭方式的参数,0-不能再读,1-不能再写,2-读写都不能。
\item[参数]		 
\begin{description}
\item[sockfd]	由socket()调用返回的套接口文件描述符
\item[how]		
\begin{description}
\item[SHUT\_{}RD(0)]	关闭sockfd上的读功能,此选项将不允许sockfd进行读操作。
\item[SHUT\_{}WR(1)]	关闭sockfd的写功能,此选项将不允许sockfd进行写操作。
\item[SHUT\_{}RDWR(2)]	关闭sockfd的读写功能。
\end{description}
\end{description}
\item[返回值]		成功则返回0,错误返回-1。
\end{description}



在远程的主机可能试图使用connect()连接
你使用listen()正在监听的端口。但此连接将会在队列中等待,直到使用accept()处理
它。调用accept()之后,将会返回一个全新的套接口文件描述符来处理这个单个的连接。这样,
对于同一个连接来说,你就有了两个文件描述符。原先的一个文件描述符正在监听你指定的端
口,新的文件描述符可以用来调用send()和recv()。


















\subsection{辅助函数}
\subsubsection{inet\_{}addr}
\begin{lstlisting}[style=C]
	sockaddr_in ina;
	ina.sin_addr.s_addr = inet_addr("132.241.5.10");
\end{lstlisting}
\begin{description}
\item[功能]	inet\_{}addr函数将IP地址从点数格式转换成无符号长整型,其返回的地址已经是网络字节格式,所以无需再调用函数htonl()。
\end{description}


\subsubsection{inet\_{}ntoa}
\begin{lstlisting}[style=C]
	sockaddr_in ina;
	printf("%s",inet_ntoa(ina.sin_addr));
\end{lstlisting}
\begin{description}
\item[功能]	inet\_ntoa函数将一个in\_{}addr结构体输出成点数格式,ntoa含义是“network to ascii”,它返回的是一个指向一个字符的指针。它是一个由inet\_{}ntoa()控制的静态的固定的指针,所以每次调用inet\_{}ntoa(),它就将覆盖上次调用时所得的IP地址。
\end{description}


\subsubsection{bzero}
\begin{lstlisting}[style=C]
#include <strings.h>
void bzero(void* s,size_t n);
\end{lstlisting}
\begin{description}
\item[功能]	设置s指向的前n个字节的空间里的值为0(包含$\backslash{}0$)。
\end{description}


\subsubsection{htons}
\begin{lstlisting}[style=C]
#include <netinet/in.h>
uint16_t htons(uint16_t hostshort);
\end{lstlisting}
\begin{description}
\item[功能]	将unsigned short整型的主机点数格式的地址转换为ip地址。一般用来转换端口。
\end{description}


\subsubsection{inet\_{}aton}
\begin{lstlisting}[style=C]
#include <sys/socket.h>
#include <netinet/in.h>
int inet_aton(const char *cp, struct in_addr *inp);
\end{lstlisting}
\begin{description}
\item[功能]	将IP地址从点数格式的形式转换成二进制,并将它存储在inp的结构种。
\item[返回值]		成功则返回0,地址非法则返回-1。
\end{description}








