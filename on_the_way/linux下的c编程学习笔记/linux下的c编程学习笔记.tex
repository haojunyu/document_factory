\documentclass[10pt,a4paper]{article}	
\usepackage{amsmath}
\usepackage{amsfonts}
\usepackage{amssymb}
\usepackage{CJK}													% 支持中文
\usepackage{listings}											% 支持代码显示
\usepackage{color}												% 给文字,表格和图形上色(68种)
\usepackage{xcolor}												% color包的扩展
\usepackage[colorlinks,citecolor = blue, linkcolor=blue,hyperindex,CJKbookmarks]{hyperref}	% 链接功能

%===========================c语言代码显示设置===============================
\lstdefinestyle{C}{
language=[ANSI]c,							% c语言
basicstyle=\small,							% 小号字体
numbers=left,								% 代码左边显行号
keywordstyle=\color{blue},					% 关键字用蓝色显示
numberstyle={\tiny\color{green}},			% 行号小小号,绿色	
numbersep=5pt,								% 行号与代码的距离
commentstyle=\small\color{red},				% 注释颜色和字号
backgroundcolor=\color[rgb]{0.95,1.0,1.0},	% 设置背景颜色
showspaces=false,							% 不显示空格
showtabs=false,								% 不显示\t
tabsize=4,									% \t的长短
frame=shadowbox, 							% 添加外框
framexleftmargin=5mm, 						% 外框左边的页边空白
rulesepcolor=\color{red!20!green!20!blue!20!},	%设置阴影颜色
breaklines=true,								% 自动断行
morekeywords={MYSQL,MYSQL_RES,MYSQL_ROW},
escapeinside=``,
extendedchars=false 							% 解决代码跨页时,章节标题,页眉等汉字不显示的问题
}
%========================SHELL代码显示设置=============================
\lstdefinestyle{BASH}{
language=bash,								% bash代码
basicstyle=\small,							% 小号字体
numbers=left,								% 代码左边显行号
keywordstyle=\color{blue},					% 关键字用蓝色显示
numberstyle={\tiny\color{green}},			% 行号小小号,绿色	
numbersep=5pt,								% 行号与代码的距离
commentstyle=\small\color{red},				% 注释颜色和字号
backgroundcolor=\color[rgb]{1.0,0.95,1.0},	% 设置背景颜色
showspaces=false,							% 不显示空格
showtabs=false,								% 不显示\t
tabsize=4,									% \t的长短
frame=shadowbox, 							% 添加外框
framexleftmargin=5mm, 						% 外框左边的页边空白
rulesepcolor=\color{red!20!green!20!blue!20!},	%设置阴影颜色
breaklines=true,								% 自动断行
escapeinside=``,
extendedchars=false, 						% 解决代码跨页时,章节标题,页眉等汉字不显示的问题
morekeywords={cp,xz,tar},
literate={\$}{{\textcolor{red}{\$}}}1
         {:}{{\textcolor{gray}{:}}}1
         {hjy@jy}{{\textcolor{orange}{hjy@jy}}}6
         {sudo}{{\textcolor{red}{sudo}}}4,
}
%========================================================================

\begin{document}

\begin{CJK*}{UTF8}{gkai}
%============================++题目和作者++================================
\title{linux下的c编程学习笔记}					   					% 题目
\author{郝俊禹\thanks{Email:haojunyu2012@gmail.com}}				% 作者
%============================++++++++++++=================================
\date{}                                             				% 显示作者,不显示日期
\maketitle                                          				% 生成标题
\tableofcontents 												% 生成目录
\clearpage

%%socket
\section{socket网络编程}
\subsection{常见结构体}
\subsubsection{sockaddr}
%\lstinputlisting[lastline=4]{main.c}
\begin{lstlisting}[style=C]
struct sockaddr {
	unsigned  short  sa_family;     	//address family="AF_XXX",often use AF_INET
	char  sa_data[14];  				//14 characters of address.            
};
\end{lstlisting}
sockaddr该结构体用作bind、connect、recvfrom、sendto等函数的参数,指明地址信息。但一般编程中不直接对此数据结构操作,而是使用与sockaddr等价的数据结构sockaddr\_{}in(netinet/in.h)

\subsubsection{sockaddr\_{}in}
\begin{lstlisting}[style=C]
struct  sockaddr_in {
	short  int  sin_family; 				/* Address family */
	unsigned  short  int  sin_port; 		/* Port number */
	struct  in_addr  sin_addr;			/* Internet address [htons(INADDR_ANY)]*/
	unsigned  char  sin_zero[8]; 		/* Same size as struct sockaddr(16 byte) */
};
\end{lstlisting}
\begin{lstlisting}[style=C]
struct  in_addr {
	unsigned  long  s_addr;				/* address in network byte order */
};
\end{lstlisting}
填值的时候使用sockaddr\_{}in结构,而作为函数的参数传入的时候转换成sockaddr结构就行了,毕竟都是16个字符长。


\subsection{主要函数}
\subsubsection{socket}
\begin{lstlisting}[style=C]
#include<sys/types.h>
#include<sys/socket.h>
int socket(int domain,int type,int protocol);
\end{lstlisting}
\begin{description}
\item[功能]	使用前创建一个新的套接字
\item[参数]		 
\begin{description}
\item[domain] 域名,一般设置为“AF\_{}INET”(ipv4)
\item[type]	套接口类型,一般为SOCK\_{}STREAM(tcp)或SOCK\_{}DGRAM(udp)
\item[protocol]	协议,一般设置为0
\end{description}
\item[返回值]		返回一个套接口描述符,如果出错,则返回­1。
\end{description}

\subsubsection{bind}
\begin{lstlisting}[style=C]
#include<sys/types.h>
#include<sys/socket.h>
int bind(int sockfd,struct sockaddr* my_addr,int addrlen);
\end{lstlisting}
\begin{description}
\item[功能]	把套接口绑定到本地计算机的某一个端口上。但如果你只想使用
connect()则无此必要。
\item[参数]		 
\begin{description}
\item[sockfd] 	由socket()调用返回的套接口文件描述符
\item[my\_{}addr]	指向数据结构sockaddr的指针,sockaddr中包括了关于你的地址、端口和IP地址的信息。
\item[addrlen]	地址长度,一般设置为sizeof(structsockaddr)。
\end{description}
\item[返回值]		没有错误,就返回0,否则返回-1
\end{description}

\subsubsection{connect}
\begin{lstlisting}[style=C]
#include<sys/types.h>
#include<sys/socket.h>
int connect(int sockfd,struct sockaddr* serv_addr,int addrlen);
\end{lstlisting}
\begin{description}
\item[功能]	完成面向连接的协议的连接过程。在建立连接的时候总会有一方先发送数据,
那么谁调用了connect谁就是先发送数据的一方。
\item[参数]		 
\begin{description}
\item[sockfd]	由socket()调用返回的套接口文件描述符
\item[serv\_{}addr]	指向数据结构sockaddr的指针,sockaddr中包括了关于你的地址、端口和IP地址的信息。
\item[addrlen]	地址长度,一般设置为sizeof(structsockaddr)。
\end{description}
\item[返回值]		没有错误,就返回0,否则返回SOCKET\_{}ERROR(-1)
\end{description}


\subsubsection{listen}
\begin{lstlisting}[style=C]
#include<sys/types.h>
#include<sys/socket.h>
int listen(int sockfd,int backlog);
\end{lstlisting}
\begin{description}
\item[功能]	用于面向连接服务器,表明它愿意接收连接.一般先调用listen(),然后再调用accept()。
\item[参数]		 
\begin{description}
\item[sockfd]	由socket()调用返回的套接口文件描述符
\item[backlog]	进入队列中允许的连接的个数。进入的连接请求在使用系统调用
accept()应答之前要在进入队列中等待。这个值是队列中最多可以拥有的请求的
个数。大多数系统的缺省设置为20。你可以设置为5或者10。
\end{description}
\item[返回值]		没有错误,就返回0,否则返回SOCKET\_{}ERROR(-1)
\end{description}


\subsubsection{accept}
\begin{lstlisting}[style=C]
#include<sys/socket.h>
int accept(int sockfd,void* addr,int* addrlen);
\end{lstlisting}
\begin{description}
\item[功能]	客户端用connect()连接,服务端用listen()监听端口。但此连接将会在队列中等待,直到服务端使用accept()处理
它。调用accept()之后,将会返回一个全新的套接口文件描述符来处理这个单个的连接。
\item[参数]		 
\begin{description}
\item[sockfd]	由socket()调用返回的套接口文件描述符
\item[addr]		addr是指向本地的数据结构sockaddr\_{}in的指针,客户端的信息存储在其中
\item[addrlen]	地址长度,一般为sizeof(structsockaddr)。
\end{description}
\item[返回值]		没有错误,就返回一个全新的套接口文件描述符,否则返回-1。
\end{description}


\subsubsection{send[tcp]}
\begin{lstlisting}[style=C]
#include<sys/socket.h>
int send(int sockfd,const void* msg,int len,int flags);
\end{lstlisting}
\begin{description}
\item[功能]	客户端发送数据
\item[参数]		 
\begin{description}
\item[sockfd]	由socket()调用返回的套接口文件描述符,可以是通过socket()
系统调用返回的,也可以是通过accept()系统调用得到的。
\item[msg]		指向你希望发送的数据的指针。
\item[len]		数据的字节长度
\item[flags]		参数标志,一般设置为0。
\end{description}
\item[返回值]		返回实际发送的字节数,这可能比你实际想要发送的字节数少。
如果返回的字节数比要发送的字节数少,你在以后必须发送剩下的数据。当
send()出错时,将返回­1。
\end{description}


\subsubsection{sendto[udp]}
\begin{lstlisting}[style=C]
#include<sys/socket.h>
ssize_t sendto(int sockfd, void *buf, size_t len, int flags,struct sockaddr *src_addr, socklen_t *addrlen);
\end{lstlisting}
\begin{description}
\item[功能]	客户端发送数据(常用于udp连接)
\item[参数]		 
\begin{description}
\item[sockfd]	标识一个已连接套接口的描述字。
\item[buff]		接收数据缓冲区。 
\item[len]		缓冲区长度。 
\item[flags]		参数标志,一般设置为0。
\item[src\_{}addr]	指向装有源地址的缓冲区。
\item[addren]	指向装有源地址的缓冲区。	
\end{description}
\item[返回值]		返回实际发送的字节数,这可能比你实际想要发送的字节数少。
如果返回的字节数比要发送的字节数少,你在以后必须发送剩下的数据。当
sendto()出错时,将返回­1。
\end{description}


\subsubsection{recv[tcp]}
\begin{lstlisting}[style=C]
#include<sys/socket.h>
int recv(int sockfd,void* buf,int len,unsigned int flags);
\end{lstlisting}
\begin{description}
\item[功能]	服务端接受数据(常用于tcp连接)
\item[参数]		 
\begin{description}
\item[sockfd]	由socket()调用返回的套接口文件描述符,可以是通过socket()
系统调用返回的,也可以是通过accept()系统调用得到的。
\item[msg]		指向你希望发送的数据的指针。
\item[len]		缓冲区的最大长度
\item[flags]		参数标志,一般设置为0。
\end{description}
\item[返回值]		返回实际读取到缓冲区的字节数,如果出错则返回­1。
\end{description}


\subsubsection{recvfrom[udp]}
\begin{lstlisting}[style=C]
#include<sys/socket.h>
ssize_t recvfrom(int sockfd, void *buf, size_t len, int flags,struct sockaddr *src_addr, socklen_t *addrlen);
\end{lstlisting}
\begin{description}
\item[功能]	服务端接受数据(常用于udp连接)
\item[参数]		 
\begin{description}
\item[sockfd]	标识一个已连接套接口的描述字。
\item[buff]		接收数据缓冲区。 
\item[len]		缓冲区长度。 
\item[flags]		参数标志,一般设置为0。
\item[src\_{}addr]	指向装有源地址的缓冲区。
\item[addren]	指向装有源地址的缓冲区。	
\end{description}
\item[返回值]		返回实际读取到缓冲区的字节数,如果出错则返回­1。
\end{description}


\subsubsection{close}
\begin{lstlisting}[style=C]
#include<unistd.h>
int close(int sockfd);
\end{lstlisting}
\begin{description}
\item[功能]	关闭本进程的socketfd,但链接还是开着的,用这个socketfd的其它进程还能用这个链接,能读或写这个socketfd。
\item[参数]		 
\begin{description}
\item[sockfd]	由socket()调用返回的套接口文件描述符,可以是通过socket()
系统调用返回的,也可以是通过accept()系统调用得到的。
\end{description}
\item[返回值]		成功则返回0,错误返回-1。
\end{description}


\subsubsection{shutdown}
\begin{lstlisting}[style=C]
#include<sys/socket.h>
int shutdown(int sockfd,int how);
\end{lstlisting}
\begin{description}
\item[功能]	破坏了socket链接,读的时候可能侦探到EOF结束符,写的时候可能会收到一个SIGPIPE信号,这个信号可能直到
socket buffer被填充了才收到,shutdown还有一个关闭方式的参数,0-不能再读,1-不能再写,2-读写都不能。
\item[参数]		 
\begin{description}
\item[sockfd]	由socket()调用返回的套接口文件描述符
\item[how]		
\begin{description}
\item[SHUT\_{}RD(0)]	关闭sockfd上的读功能,此选项将不允许sockfd进行读操作。
\item[SHUT\_{}WR(1)]	关闭sockfd的写功能,此选项将不允许sockfd进行写操作。
\item[SHUT\_{}RDWR(2)]	关闭sockfd的读写功能。
\end{description}
\end{description}
\item[返回值]		成功则返回0,错误返回-1。
\end{description}



在远程的主机可能试图使用connect()连接
你使用listen()正在监听的端口。但此连接将会在队列中等待,直到使用accept()处理
它。调用accept()之后,将会返回一个全新的套接口文件描述符来处理这个单个的连接。这样,
对于同一个连接来说,你就有了两个文件描述符。原先的一个文件描述符正在监听你指定的端
口,新的文件描述符可以用来调用send()和recv()。


















\subsection{辅助函数}
\subsubsection{inet\_{}addr}
\begin{lstlisting}[style=C]
	sockaddr_in ina;
	ina.sin_addr.s_addr = inet_addr("132.241.5.10");
\end{lstlisting}
\begin{description}
\item[功能]	inet\_{}addr函数将IP地址从点数格式转换成无符号长整型,其返回的地址已经是网络字节格式,所以无需再调用函数htonl()。
\end{description}


\subsubsection{inet\_{}ntoa}
\begin{lstlisting}[style=C]
	sockaddr_in ina;
	printf("%s",inet_ntoa(ina.sin_addr));
\end{lstlisting}
\begin{description}
\item[功能]	inet\_ntoa函数将一个in\_{}addr结构体输出成点数格式,ntoa含义是“network to ascii”,它返回的是一个指向一个字符的指针。它是一个由inet\_{}ntoa()控制的静态的固定的指针,所以每次调用inet\_{}ntoa(),它就将覆盖上次调用时所得的IP地址。
\end{description}


\subsubsection{bzero}
\begin{lstlisting}[style=C]
#include <strings.h>
void bzero(void* s,size_t n);
\end{lstlisting}
\begin{description}
\item[功能]	设置s指向的前n个字节的空间里的值为0(包含$\backslash{}0$)。
\end{description}


\subsubsection{htons}
\begin{lstlisting}[style=C]
#include <netinet/in.h>
uint16_t htons(uint16_t hostshort);
\end{lstlisting}
\begin{description}
\item[功能]	将unsigned short整型的主机点数格式的地址转换为ip地址。一般用来转换端口。
\end{description}


\subsubsection{inet\_{}aton}
\begin{lstlisting}[style=C]
#include <sys/socket.h>
#include <netinet/in.h>
int inet_aton(const char *cp, struct in_addr *inp);
\end{lstlisting}
\begin{description}
\item[功能]	将IP地址从点数格式的形式转换成二进制,并将它存储在inp的结构种。
\item[返回值]		成功则返回0,地址非法则返回-1。
\end{description}










%mysql
\section{mysql的使用心得}
\subsection{mysql命令}
\subsubsection{用户管理}
\begin{description}
\item[启动服务]	service mysql start(ubuntu) 
\item[登录]	mysql -u $username$ -p$password$
\item[退出]	exit	
\item[创建用户]	insert into mysql.user(Host,User,Password) values("$localhost$","$username$","password($password$)")
\item[给用户授予数据库的全部权限]	grant all privileges on $dbname$.$*$ to $username$@$localhost$ identified by '$password$' 
\item[给用户授予数据库的部分权限]	grant insert,delete,update,select on $dbname$.$tablename$ to $username$@$localhost$ identified by '$password$' 
\item[刷新系统权限表]	flush privileges
\item[删除用户]	delete * from mysql.user where User="$username$" and Host="$localhost$"
\item[修改用户密码]	update mysql.user set password=password("$password$") where User="$username$" and Host="$localhost$"
\end{description}


\subsubsection{数据库管理}
\begin{description}
\item[备份数据库]		mysqldump -u $username$ -p$password$ $dbname$ [$tablename$] $>$ $C:\setminus{}file.sql$
\item[导入数据库]		mysql -u $username$ -p$password$ $dbname$ $<$ $C:\setminus{}file.sql$
\item[执行sql脚本]	source $createdb.sql$
\end{description}


\subsubsection{操作数据库}
\begin{itemize}
\item 增
\begin{description}
\item[建立数据库]	create database $dbname$;	
\item[建立表]		create table if not exists $tablename$(userId INT, userName VARCHAR(50));
\item[复制表]		create table $tablename2$ select * from $tablename1$ where 1 <> 1 
\item[创建(唯一)索引]	create [unique] index $idxId$ on $tablename$($colname$)
\end{description}

\item 删
\begin{description}
\item[数据库]		drop database $dbname$
\item[数据库中的表]	drop table $tablename$
\item[清空表中记录]	delete from $tablename$
\item[删除索引]		drop index $idxId$ on $tablename$
\end{description}

\item 改
\begin{description}
\item[修改(增加)多个字段]	alter $tablename$ add column $fieldname1$ $datetype1$,add column $fieldname2$ $datatype2$
\item[修改字段类型]	alter table $tablename$ modify $colname$ $coltype$
\item[修改字段名称以及类型]		alter tabel $tablename$ change $colname$ $newcolname$ $datatype$
\item[更行表中记录]	update $tablename$ set $userName$=$'hjy'$ where $userId$=$'0005'$
\end{description}

\item 查
\begin{description}
\item[显示数据库]		show databases
\item[使用数据库]		use $dbname$
\item[显示表格]		show tables
\item[显示表格列的属性]	show columns from $tablename$
\item[查询时间]	select now()
\item[查询当前用户]	select user()
\item[查询数据库版本]	select version()
\item[查询当前使用的数据库]	select database()
\end{description}
\end{itemize}


\subsubsection{内置函数}
\begin{description}
\item[合并字段显示]	select concat($userId,':',userName,"="$) from $tablename$
\item[选择10-20行]	select * from $tablename$ order by $colname$ limit $9$,$10$
\end{description}



\subsection{C访问和操作mysql}
\subsubsection{准备(ubuntu)}
\begin{itemize}
\item 安装mysql-server
\begin{lstlisting}[style=BASH]
hjy@jy:~$ sudo apt-get install mysql-server-5.5
\end{lstlisting}

\item 安装用于连接到数据库和执行数据库查询的库文件mysqlclient.
\begin{lstlisting}[style=BASH]
hjy@jy:~$ sudo apt-get install libmysqlclient-dev
\end{lstlisting}
安装成功后,相关文件如下:
\begin{description}
\item[头文件(mysql.h)] 位于/usr/include/mysql目录下;
\item[库文件(libmysqlclient.so)] 位于/usr/lib/mysql和/usr/lib目录下
\end{description}
\end{itemize}


\subsubsection{操作数据库}
\begin{itemize}
\item 控制类函数
\begin{description}
\item[mysql\_{}init]	初始化MYSQL对象
\item[mysql\_{}options]	设置连接选项
\item[mysql\_{}real\_{}connect]	连接到mysql数据库
\item[mysql\_{}real\_{}escape\_{}string]	将查询串合法化
\item[mysql\_{}query]	发出一个以空字符结束的查询串
\item[mysql\_{}real\_{}query]	发出一个查询串
\item[mysql\_{}store\_{}result]	一次性传送结果
\item[mysql\_{}use\_{}result]	逐行传送结果
\item[mysql\_{}free\_{}result]	释放结果集
\item[mysql\_{}change\_{}user]	改变用户
\item[mysql\_{}select\_{}db]	改变默认数据库
\item[mysql\_{}debug]	送出调试信息
\item[mysql\_{}dump\_{}debug\_{}info]	转储调试信息
\item[mysql\_{}ping]	测试数据库是否处于活动状态
\item[mysql\_{}shutdown]	请求数据库SHUTDOWN
\item[mysql\_{}close]	关闭数据库连接	
\end{description}

\item 信息获取类函数
\begin{description}
\item[mysql\_{}character\_{}set\_{}name]	获取默认字符集
\item[mysql\_{}get\_{}client\_{}info]	获取客户端信息
\item[mysql\_{}host\_{}info]	获取主机信息
\item[mysql\_{}get\_{}proto\_{}info]	获取协议信息
\item[mysql\_{}get\_{}server\_{}info]	获取服务器信息
\item[mysql\_{}info]	获取部分查询语句的附加信息
\item[mysql\_{}stat]	获取数据库状态
\item[mysql\_{}list\_{}dbs]	获取数据库列表
\item[mysql\_{}list\_{}tables]	获取数据表列表
\item[mysql\_{}list\_{}fields]	获取字段列表
\end{description}

\item 行类类操作函数
\begin{description}
\item[mysql\_{}field\_{}count]	获取字段数
\item[mysql\_{}affected\_{}rows] 获取受影响的行数
\item[mysql\_{}insert\_{}id]	获取AUTO\_{}INCREMENT列的ID值
\item[mysql\_{}num\_{}fields]	获取结果集中的字段数
\item[mysql\_{}field\_{}tell]	获取当前字段位置
\item[mysql\_{}field\_{}seek]	定位字段
\item[mysql\_{}fetch\_{}field]	获取当前字段
\item[mysql\_{}fetch\_{}field\_{}direct]	获取指定字段
\item[mysql\_{}fetch\_{}fields]	获取所有字段的数组
\item[mysql\_{}num\_{}rows]	获取行数
\item[mysql\_{}fetch\_{}length]	获取行长度
\item[mysql\_{}row\_{}tell]	获取当前行位置
\item[mysql\_{}row\_{}seek]	行定位
\item[mysql\_{}data\_{}seek]	行定位
\item[mysql\_{}fetch\_{}row]	获取当前行
\end{description}

\item 线程类操作函数
\begin{description}
\item[mysql\_{}list\_{}processes]	返回所有线程列表
\item[mysql\_{}thread\_{}id]	返回当前线程ID
\item[mysql\_{}thread\_{}safe]	是否支持线程方式
\item[mysql\_{}kill]	杀死一个线程
\end{description}

\item 出错处理类函数
\begin{description}
\item[mysql\_{}errno]	获取错误号
\item[mysql\_{}error]	获取错误信息
\end{description}

\item c程序样例
\begin{lstlisting}[style=C]
#include <stdio.h>
#include <string.h>
#include <mysql.h>

int main()
{
	MYSQL mysql;
	MYSQL_RES *res;
	MYSQL_ROW row;
	char sql[50];
	int t;
	
	sprintf(sql,"select * from USER"); 	// 定义执行的SQL语句
	mysql_init(&mysql); 				//初始化mysql结构
	if(!mysql_real_connect(&mysql,"localhost","virtual_desktop","virtual_desktop","virtual_desktop",3306,NULL,0))
		printf("\n数据库连接发生错误:%s\n",mysql_error(&mysql));
	else
		printf("\n数据库连接成功!\n");
		
	t = mysql_real_query(&mysql, sql, (unsigned int)strlen(sql));
								//在服务器上执行定义的SQL语句
	if(t)
		printf("查询语句错误: %s\n",mysql_error(&mysql));
	else
	{
		res = mysql_store_result(&mysql);
		while(row = mysql_fetch_row(res))
		{
			for(t = 0; t < mysql_num_fields(res); t++)
			printf("%s",row[t]);
			printf("\n");
		}
	}
	printf("释放结果集的空间...\n");
	mysql_free_result(res);
	
	mysql_close(&mysql); 				//释放数据库
	printf("sql errror! %s\n",mysql_error(&mysql));
	
	return 0;
}
\end{lstlisting}
\end{itemize}


\subsubsection{编译\&{}连接}
\begin{itemize}
\item 编译
\begin{lstlisting}[style=BASH]
hjy@jy:~$ gcc -c mysql.c -I /usr/include/mysql
\end{lstlisting}
-I 指定头文件目录

\item 连接
\begin{lstlisting}[style=BASH]
hjy@jy:~$ gcc mysql.o -o main -L /usr/lib -lmysqlclient
\end{lstlisting}
-L 指定库文件路径\\
-lmysqlclient libmysqlclient.so库文件

\textbf{\underline{库文件类型简介}}
\begin{description}
\item[.o]	编译的目标文件
\item[.a]	静态库,其实就是把若干.o文件打了个包
\item[.so]	动态链接库(共享库)
\item[.lo]	使用libtool编译出来的目标文件,其实就是在.o文件中添加了一些信息
\item[.la]	使用libtool编译出来的库文件,其实就是个文本文件,记录同名动态库和静态库的相关信息
\end{description}

\end{itemize}


\clearpage     
\end{CJK*}
\end{document}