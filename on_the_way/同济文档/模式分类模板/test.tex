\documentclass[12pt,a4paper,CJK]{beamer}
\usepackage[utf8]{inputenc}
\usepackage{amsmath}
\usepackage{amsfonts}
\usepackage{amssymb}
\usepackage{CJK}										% 支持中文
\usepackage{color}						% 给文字,表格和图形上色(68种)
\usepackage{xcolor}									% color包的扩展

%================================主题设置=============================%
\usetheme{Boadilla}									% PPT主题
\usefonttheme{serif} 
\useinnertheme{umbcboxes}


%================================具体设置=============================%
\setbeamercolor{umbcboxes}{bg=violet!12,fg=black}
\setbeamertemplate{navigation symbols}{}				% 取消引导
\hypersetup{colorlinks,citecolor = blue, linkcolor=blue}
\DeclareMathOperator*{\argmax}{arg\,max}
	
%============================++++++++++++============================%
\author{hjy}
\title{基于KFDA}
\institute{TongJi University}
\date{\today}	
%============================++++++++++++============================%	



\begin{document}
\begin{CJK*}{UTF8}{gkai}
%----------- titlepage ----------------------------------------------%
\begin{frame} 				
	\titlepage 
\end{frame}
	
%----------- outline ------------------------------------------------%
\begin{frame}
	\frametitle{目录}
	\tableofcontents
\end{frame}



%--------------------------------------------------------------------%
\section{核函数定义}
\begin{frame}{\secname}
		选择使得 Fisher准则函数达到极值的矢量作为最佳投影方向$\overrightarrow{W}$,从而使得样本在该方向上投影后,达到最大的类间离散度$S_b$和最小的类内离散度$S_w$.
		
		
		
		
\end{frame}



\end{CJK*}
\end{document}