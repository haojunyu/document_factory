\part{Linux}
\setcounter{section}{0}
\clearpage
\section{压缩刻录}
\subsection{unetbootin-制作usb启动盘}
准备工作
\begin{itemize}
\item u盘
\item iso镜像文件--\href{http://releases.ubuntu.com/precise/}{ubuntu-12.04.2-desktop-amd64.iso}
\item 刻录软件--\href{http://www.oschina.net/news/9166/oschina-os-week-recommended-UNetbootin}{unetbootin-linux-583}
\end{itemize} 
操作步骤
\begin{enumerate}
\item 格式化u盘
\begin{itemize}
\item 查看u盘信息
\begin{lstlisting}[style=BASH]
hjy@jy:~$ mount
/dev/sdb1 on /media/KINGSTON type vfat (rw,nosuid,nodev,uid=1000,gid=1000,shortname=mixed,dmask=0077,utf8=1,showexec,flush,uhelper=udisks)
\end{lstlisting}

\item 卸载u盘
\begin{lstlisting}[style=BASH]
hjy@jy:~$ umount /media/KINGSTON
\end{lstlisting}

\item 格式化
\begin{lstlisting}[style=BASH]
hjy@jy:~$ mkfs -t vfat /dev/sdb1
\end{lstlisting}
\end{itemize}
\item 启动并设置unetbootin-linux-583参数\\
\begin{figure}[!htbp]
	\centering
	\caption{参数设置}  
		\includegraphics[scale=0.35]{figs/ubuntu_unetbootin_set.pdf}
    	\label{fig:unetbootin_set}
\end{figure}
注意:\textcolor{red}{软件能够识别u盘--u盘一定要插在电脑上}
\end{enumerate}
\textcolor{blue}{ubuntu安装命令:}
\begin{lstlisting}[style=BASH]
hjy@jy:~$ sudo apt-get install unetbootin
\end{lstlisting}

\clearpage
\section{聊天工具}
\section{视频软件}
\subsection{xbmc--包罗万象的媒体中心}
准备工作
\begin{itemize}
\item 插件--\href{}{xbmc中文插件.zip}
\end{itemize}

安装步骤:
\begin{enumerate}
\item 添加软件源
\begin{lstlisting}[style=BASH]
hjy@jy:~$ sudo add-apt-repository ppa:team-xbmc/ppa
\end{lstlisting}

\item 更新软件源
\begin{lstlisting}[style=BASH]
hjy@jy:~$ sudo apt-get update
\end{lstlisting}

\item 安装软件
\item 更新软件源
\begin{lstlisting}[style=BASH]
hjy@jy:~$ sudo apt-get install xbmc
\end{lstlisting}
\end{enumerate}
软件配置:
\begin{itemize}
\item 汉化
\begin{enumerate}
\item System$\rightarrow{}$appearance$\rightarrow{}$Skin$\rightarrow{}$Arial based\textcolor{red}{(必须先设置)}
\item International$\rightarrow{}$Chinese(simple),Character Set$\rightarrow{}$Chinese Simplified
\end{enumerate}
\item 插件
\begin{enumerate}
\item 解压出repository.googlecode.xbmc-addons-chinese.zip\textcolor{red}{(就是zip格式)}
\item 系统设置$\rightarrow{}$扩展功能$\rightarrow{}$从zip中安装
\item 从扩展功能里面选择获取扩展功能,点全部扩展功能
\begin{itemize}
\item 视频插件
\begin{itemize}
\item 迅雷看看(KanKan)
\item PPTV视频
\item 酷狗MV(KugouMV)
\item 乐视网(LeTV)
\item 优酷视频(YouKu)
\item PPS网络电视(PPStream)
\item 中国网络电视台点播(CNTV)
\item 音悦台MV(YinYueTai)
\item 土豆视频(Tudou)
\item 奇艺视频(QIYI)
\item 腾讯视频(Tecent)
\item 新浪视频(Sina)
\end{itemize}
\item 音频插件
\begin{itemize}
\item KuwoBox/酷我音乐盒插件
\item BaiduRadio/百度电台联盟插件
\item TOP100/巨鲸音乐网插件
\end{itemize}
\item 字幕
\begin{itemize}
\item LRC同步歌词脚本
\item shooterSub/射手字幕脚本
\end{itemize}
\item 天气
\begin{itemize}
\item 中国天气(Weather China)
\end{itemize}
\end{itemize}
\end{enumerate}
\end{itemize}





\clearpage
\section{音乐软件}
\section{游戏软件}
\section{网络游戏}
\section{浏览器}
\section{图形图像}
\section{输入法}
\section{下载工具}
\section{办公软件}
\subsection{latex(包括中文环境)}
\subsubsection{精简版}

\begin{itemize}
\item 安装texlive-latex-base:基本核心包
\begin{lstlisting}[style=BASH]
hjy@jy:~$ sudo apt-get install texlive-latex-base
\end{lstlisting}
\item 安装latex-cjk-all:中文环境需要的包
\begin{lstlisting}[style=BASH]
hjy@jy:~$ sudo apt-get install texlive-cjk-all
\end{lstlisting}
\item 安装texlive-latex-extra:一些常用的包文件
\begin{lstlisting}[style=BASH]
hjy@jy:~$ sudo apt-get install texlive-latex-extra
\end{lstlisting}
\item 安装texmaker:latex文档编辑器
\begin{lstlisting}[style=BASH]
hjy@jy:~$ sudo apt-get install texmaker
\end{lstlisting}
\end{itemize}

注意:\textcolor{red}{有些.sty文件可能没有安装,如lastpage.sty.可按下面的命令来查找相应的包}
\begin{lstlisting}[style=BASH]
hjy@jy:~$ apt-cache search lastpage
\end{lstlisting}

\subsubsection{完整版}

\begin{itemize}
\item 安装texlive-full:texlive完整包
\begin{lstlisting}[style=BASH]
hjy@jy:~$ sudo apt-get install texlive-full
\end{lstlisting}
\item 安装texmaker:latex文档编辑器
\begin{lstlisting}[style=BASH]
hjy@jy:~$ sudo apt-get install texmaker
\end{lstlisting}
\end{itemize}
注意:\textcolor{red}{完整包比较大,下载可能需要不少时间}

\subsubsection{\href{http://blog.csdn.net/sdupine/article/details/7958668}{添加utf8字体}}

\begin{itemize}
\item 首先要生成Tex用户个人配置,使用Tex的updmap 命令
\begin{lstlisting}[style=BASH]
hjy@jy:~$ updmap
hjy@jy:~$ sudo update-updmap
hjy@jy:~$ updmap
\end{lstlisting}
\textcolor{red}{Latex默认的用户配置文件位于Home下的.texmf-var [~/.texmf-var]目录中,请检查命令给出的信息,确定目录,后面会使用。}
\item 给Tex用户个人配置目录添加读写权限
\begin{lstlisting}[style=BASH]
hjy@jy:~$ sudo chmod -R 777 .texmf-var
\end{lstlisting}
\item 修改mkfont文件夹中的mkfont.sh文件\\
将第四行修改为:
\begin{lstlisting}[style=BASH]
TEXMF=~/.texmf-var
\end{lstlisting}
\textcolor{red}{mkfont文件夹中mkfont.sh文件中已经做过修改,要是用mkfont.tar中的文件则还要进行其它修改}
\item 把字体文件拷贝到 mkfont 目录中,并执行命令
\begin{lstlisting}[style=BASH]
hjy@jy:~$ ./mkfont.sh fontName.ttf fontName fontShortName
\end{lstlisting}
\item 测试结果如图\ref{fig:latex_fonts}
\begin{figure}[!htbp]
	\centering
	\caption{基本设置}  
		\includegraphics[scale=0.25]{figs/ubuntu_latex_fonts.png}
    	\label{fig:latex_fonts}
\end{figure}
\end{itemize}




\subsection{blogilo}
blogilo是linux下面一款开源的写博客的客户端,可以写博客园以及wordpress的博客,与windows下面的windows live有的一拼。安装命令如下:
\begin{lstlisting}[style=BASH]
hjy@jy:~$ sudo apt-get install blogilo
\end{lstlisting}

软件配置:
\begin{enumerate}
\item 博客$\rightarrow{}$添加博客,得到下图\ref{fig:blogilo_base}:
\begin{figure}[!htbp]
	\centering
	\caption{基本设置}  
		\includegraphics[scale=0.25]{figs/ubuntu_blogilo_base.pdf}
    	\label{fig:blogilo_base}
\end{figure}
在URL里输入你的博客地址,格式为“\textcolor{red}{http://www.cnblogs.com/Blog名/services/metaweblog.aspx}”,然后再输入用户名和密码。

\item 点击上图中“高级”页面,选择API为“MetaWeblog API”,然后点击“获取ID”,即可得到下图\ref{fig:blogilo_advantage}:
\begin{figure}[!htbp]
	\centering
	\caption{高级设置}  
		\includegraphics[scale=0.25]{figs/ubuntu_blogilo_advantage.pdf}
    	\label{fig:blogilo_advantage}
\end{figure}

\item 单击确定,大功告成。

\end{enumerate}




\clearpage
\section{阅读翻译}
\section{系统工具}
\section{编程开发}
\subsection{matlab2011a}
准备工作\\
\begin{itemize}
\item iso镜像文件--\href{http://url.cn/Ius1xq}{Matlab.R2011a.UNIX.ISO-TBE.iso}
\item 创建挂载目录,并挂载iso文件到该目录(文件在桌面上)
\begin{lstlisting}[style=BASH]
hjy@jy:~$ sudo mkdir /mnt/iso
hjy@jy:~$ sudo mount -o loop 桌面/Matlab.R2011a.UNIX.ISO-TBE.iso /mnt/iso
\end{lstlisting}
\end{itemize} 
安装步骤\\
\begin{itemize}
\item 变更目录,并运行install进入图形安装
\begin{lstlisting}[style=BASH]
hjy@jy:~$ cd /mnt/iso
hjy@jy:~$ sudo ./install
\end{lstlisting}
\item 选择"install manually without using the internet"
\item accept the terms of the license agreement,选择“yes”
\item 选择I have the File Installation Key for my license:$59327-00840-06743-08309-05690$(standalone模式)或$31996-44762-21423-39948-52406$(network模式)
\item 选择Typical(典型安装)
\item 安装到指定目录/opt/matlab(默认会安装到/usr/local/MATLAB/R2011b文件夹中)
\item 等待、安装...
\item 激活--将crack文件夹复制到安装目录/opt/matlab
\begin{lstlisting}[style=BASH]
hjy@jy:~$ sudo cp -r crack /opt/matlab
\end{lstlisting}
若选择standalone模式,则选择"/opt/matlab/crack/license\_{}standalone.dat"文件。\\
若选择network模式,则选择"/opt/matlab/crack/license\_{}server.dat"文件。\\
\end{itemize} 

\textcolor{red}{常见问题:}
\begin{enumerate}
\item 直接输入matlab无法启动
\begin{lstlisting}[style=BASH]
hjy@jy:~$ sudo ln -s /opt/matlab/bin/matlab /usr/bin/matlab
\end{lstlisting}

\item 创建桌面启动图标
通过“启动应用程序”创建matlab.desktop。注意启动命令为\textcolor{red}{/opt/matlab/bin/matlab -desktop}

\item 在终端下运行matlab出现 /lib64/libc.so.6: not found
\begin{lstlisting}[style=BASH]
hjy@jy:~$ sudo ln -s /lib/x86_64-linux-gnu/libc.so.6 /lib64/libc.so.6
\end{lstlisting}

\item matlab下中文无法显示(中文显示方框框)
打开matlab,file->Preference->Fonts,在Desktop code font和Desktoop text font选择下拉菜单。最后面有一些无法显示名字的字体,选择它们,点击Apply就可以显示中文
\end{enumerate}


\subsection{github}
Git是一个免费的、分布式的版本控制工具,或是一个强调了速度快的源代码管理工具。每一个Git的工作目录都是一个完全独立的代码库,并拥有完整的历史记录和版本追踪能力,不依赖于网络和中心服务器。其代码状态转换图如图\ref{fig:ubuntu_git_flow}所示:
\begin{figure}[!htbp]
	\centering
	\caption{git工作状态转换图}  
		\includegraphics[scale=0.40]{figs/ubuntu_git_flow.png}
    	\label{fig:ubuntu_git_flow}
\end{figure}\\
其中各个状态注释如下:
\begin{enumerate}
\item 未被Git跟踪的状态为unstage状态,包括两种情况
\begin{itemize}
\item untrack files是指尚未被git所管理的文件
\item changed but not updated是指文件被git管理,并且发生了改变,但改动还没被git管理
\end{itemize}
\item 已经被Git跟踪的状态为stage状态,因此包括staging状态和staged状态
\item changes to be commited是指进入stage状态的文件,stage是commit和未管理之间的一个状态,也有别名叫index状态,也就是git已经管理了这些改动,但是还没完成提交。
\item .gitignore中的文件,不会出现在以上三个状态中。
\end{enumerate}

安装Git客户端如下:
\begin{lstlisting}[style=BASH]
hjy@jy:~$ sudo apt-get install git
\end{lstlisting}

软件配置:
\begin{enumerate}
\item 在https://github.com/注册一个账户(代码是托管在GitHub上),创建一个repository(ParallelComputation),SSH Keys$\rightarrow{}$ADD SSH key
\item 创建公钥
\begin{lstlisting}[style=BASH]
hjy@jy:~$ ssh-keygen -C "haojunyu2012@gmail.com" -f ~/.ssh/github
\end{lstlisting}
\item 将~/.ssh/github.pub文件中的内容复制到key中,并起一个tittle。
\item 验证方法
\begin{lstlisting}[style=BASH]
hjy@jy:~$ ssh -T git@github.com
Hi haojunyu! You've successfully authenticated, but GitHub does not provide shell access.
\end{lstlisting}
\item 设置git全局环境并查看
\begin{lstlisting}[style=BASH]
hjy@jy:~$ git config --global user.name "haojunyu"
hjy@jy:~$ git config --global user.email "haojunyu2012@gmail.com"
hjy@jy:~$ git config --list
\end{lstlisting}
\item 从服务器下载代码,准确的说应该是从GitHub服务器复制一个版本库到本地
\begin{lstlisting}[style=BASH]
hjy@jy:~$ mkdir git
hjy@jy:~$ cd git
hjy@jy:~$ git clone git@github.com:haojunyu/ParallelComputation.git
\end{lstlisting}
\item 获取到源码之后,就可以进行开发了,代码开发完成就可以提交代码
\begin{lstlisting}[style=BASH]
hjy@jy:~$ git add . 	// 往暂存区域添加已添加和修改的文件,不处理删除的文件
hjy@jy:~$ git status	// 比较本地数据目录与暂存区域的变化	
hjy@jy:~$ git commit -m "commit directions"	// 提到代码到本地数据目录,并添加提交说明
\end{lstlisting}
\item 有可能你和其他人改的是同一个文件,那么冲突的情况是在所难免的,那么在提交之后再获取一下代码,就会提示代码冲突的文件,我们需要做的就是处理这些冲突,并再次提交
\begin{lstlisting}[style=BASH]
hjy@jy:~$ git pull	// 更新代码
hjy@jy:~$ git status	
hjy@jy:~$ git commit -m "commit directions"	
\end{lstlisting}
\item 当你做完以上的步骤的时候,你需要做的是把本地数据目录的版本库的数据同步到GitHub服务器上去,这样你的同事才能够看到你作出的修改
\begin{lstlisting}[style=BASH]
hjy@jy:~$ git remote add ParaComp git@github.com:haojunyu/ParallelComputation.git	// 给仓库命名
hjy@jy:~$ git push ParaComp master		// 同步到服务器上
\end{lstlisting}

\end{enumerate}


\subsection{opengl}
OpenGL(全写Open Graphics Library)是个定义了一个跨编程语言、跨平台的编程接口的规格,它用于三维图象(二维的亦可)。OpenGL是个专业的图形程序接口,是一个功能强大,调用方便的底层图形库。

ubuntu配置opengl如下:
\begin{enumerate}
\item 安装编译环境
\begin{lstlisting}[style=BASH]
hjy@jy:~$ sudo apt-get install build-essential
\end{lstlisting}

\item 安装OpenGL Utilities
\begin{lstlisting}[style=BASH]
hjy@jy:~$ sudo apt-get install libgl1-mesa-dev
\end{lstlisting}

\item 安装OpenGL Utilities\\
一组建构于 OpenGL Library 之上的工具组,提供许多很方便的函式,使 OpenGL 更强大且更容易使用。
\begin{lstlisting}[style=BASH]
hjy@jy:~$ sudo apt-get install libgl1-mesa-dev
\end{lstlisting}

\item 安装OpenGL Utility Toolkit\\
建立在 OpenGL Utilities 上面的工具箱,除了强化了 OpenGL Utilities 的不足之外,也增加了 OpenGL 对于视窗介面支援。
\begin{lstlisting}[style=BASH]
hjy@jy:~$ sudo apt-get install freeglut3-dev
\end{lstlisting}

\end{enumerate}


\clearpage
\section{其他软件}
\subsection{ubuntu one-网盘}
背景介绍\\

Ubuntu One是由Ubuntu背后的公司Canonical所推出的一项网络服务。该服务能够存储你的文件,并允许你在多台电脑上同步,还可以与好友分享这些文件。\\
准备工作
\begin{itemize}
\item 帐号--\href{https://one.ubuntu.com/dashboard}{ubuntu one官网}
\item ubuntu系统自带软件--\href{https://one.ubuntu.com/downloads/ubuntu}{Ubuntu One}
\end{itemize} 
优势总结
\begin{description}
\item[跨平台] 目前的客户端支持windows,ubuntu,mac,android和iphone
\item[容量] 免费5G,\$2.99/月即可获得20G。
\item[速度] 同步速度大约在80kb/s,更重要的是同步响应非常快。
\item[共享] 指定欲分享用户的Email地址既可分享文件夹,发送链接既可分享指定文件。
\end{description}
使用推荐:\textcolor{red}{存储私人关键文件}

\subsection{google driver-网盘}
背景介绍\\

Google Drive,美国谷歌公司于2012年4月24日正式推出的一项云存储服务,可以向用户提供5GB的免费存储空间,同时还可以付费扩容。\\
准备工作
\begin{itemize}
\item google帐号--\href{https://accounts.google.com/SignUp?hl=zh-CN}{帐号注册}
\item ubuntu软件--\href{https://www.insynchq.com/linux}{insync}
\end{itemize} 
优势总结
\begin{description}
\item[跨平台] 目前的客户端支持windows,mac,android和ubuntu(有适用期)
\item[容量] 免费5G,\$2.49/月即可获得25G。
\item[在线浏览] 支持多达30多种文件的直接预览。
\item[集中管理] 对于文件的管理非常方便。
\end{description}
使用推荐:\textcolor{red}{作为公开网盘,用来存储文档以及常用软件}


\subsection{dropbox-网盘}
背景介绍\\

Dropbox是一个提供同步本地文件的网络存储在线应用。支持在多台电脑多种操作中自动同步。并可当作大容量的网络硬盘使用。\\
准备工作
\begin{itemize}
\item 帐号--\href{https://www.dropbox.com/}{dropbox官网}
\item 软件--ubuntu软件中心/\href{https://www.dropbox.com/install2}{dropbox}
\end{itemize} 
优势总结
\begin{description}
\item[跨平台] 目前的客户端支持windows,ubuntu,mac,android,iphone和blackberry
\item[容量] 免费3G,通过邀请朋友可获得额外空间。
\item[图片预览] 支持图片直接预览。
\item[国外流行] 成为很多软件首选的备份点,如springseed。
\end{description}
使用推荐:\textcolor{red}{配合android版备份手机上的照片}


\clearpage
\section{安全杀毒}















