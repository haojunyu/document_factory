%menu
\section{工具[tools]}
\subsection{颜色工具[color tools]}
\subsubsection{色阶[levels]}
\begin{description}
\item[定义]	表示图像亮度强弱的指数标准,也就是我们说的色彩指数,在数字图像处理教程中,指的是灰度分辨率(又称为灰度级分辨率或者幅度分辨率)。图像的色彩丰满度和精细度是由色阶决定的。色阶指亮度和颜色无关,但最亮的只有白色,最不亮的只有黑色。
\item[用途]	表现了一副图的明暗关系.如8位色的RGB空间数字图像,分别用256个阶度表示红绿蓝.
\item[用例]	
\end{description}


\subsubsection{亮度-对比度[brightness-contrast]}
\begin{description}
\item[亮度定义]	指画面的明亮程度
\item[对比度]		是一幅图像中明暗区域最亮的白和最暗的黑之间不同亮度层级的测量,差异范围越大代表对比越大,差异范围越小代表对比越小
\item[用途]	调节图像的明亮程度以及对比度
\item[用例]	
\end{description}


\subsection{其他工具[other tools]}
\subsubsection{路径[paths]}
路径工具可以创建复杂的选区,比如贝兹曲线,它有点像套索,但对所有的矢量曲线都适用。您可以编辑你的曲线,也可以用曲线画画,甚至是保存,导入,导出曲线。您还可以使用路径来创建几何图形。


\clearpage

\section{图像[Image]}
\subsection{复制图像[Duplicate]} 
该命令创建一个新的图像,这是当前图像的一个副本,包含了图像所有图层,通道和路径。 GIMP的剪贴板和历史记录都不会受到影响。

\subsection{平整图像[Flatten Image]}
平整图像命令合并所有图层并将其变成一个没有alpha通道层的单一图像层。图像平整后,它具有和之前相同的外观。所不同的是,所有的图像内容是在一个单一的没有的不透明度的图层。如果透过原始图像的所有图层有任何透明区域,那么那区域的背景颜色是可见的。

此操作使图像的结构发生显着的变化。它通常是当你想要以一种不支持图层和透明(alpha通道)的形式保存时才会用到。

\clearpage


\section{图层[Layer]}
\subsection{透明[Transparency]}
\subsubsection{透明图层到选取[alpha to selection]} 
该命令根据当前图层的alpha通道创建一个选区。不透明区域完全选中,透明区域则不会选中,半透明区域则部分被选中。这种选区将替换现有的选区。 alpha通道本身不会被更改。

在这组操作中其他的命令是相似的,除了完全替换现存的选区,他也可以将两个选取相加,相减或者求两个选区的交集。

\subsection{平整图像[Flatten Image]}
平整图像命令合并所有图层并将其变成一个没有alpha通道层的单一图像层。图像平整后,它具有和之前相同的外观。所不同的是,所有的图像内容是在一个单一的没有的不透明度的图层。如果透过原始图像的所有图层有任何透明区域,那么那区域的背景颜色是可见的。

此操作使图像的结构发生显着的变化。它通常是当你想要以一种不支持图层和透明(alpha通道)的形式保存时才会用到。

\clearpage