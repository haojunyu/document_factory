% Sample LaTeX file for creating a paper in the Morgan Kaufmannn two 
% column, 8 1/2 by 11 inch proceedings format. 

\documentstyle[proceed]{article} 

\title{Instructions for Authors} 

%\author{} % LEAVE BLANK FOR ORIGINAL SUBMISSION.
          % UAI 2010 reviewing is double-blind.

% The author names and affiliations should appear only in the accepted paper.
%
\author{ {\bf Harry Q.~Bovik\thanks{Footnote for author to give an 
alternate address.}} \\  
Computer Science Dept. \\  
Cranberry University\\ 
Pittsburgh, PA 15213 \\ 
\And 
{\bf Coauthor}  \\ 
Affiliation          \\ 
Address \\              
\And 
{\bf Coauthor}   \\ 
Affiliation \\          
Address    \\           
(if needed)\\ 
} 
 
\begin{document} 
 
\maketitle 
 
\begin{abstract} 
The Abstract paragraph should be indented 0.25 inch (1.5 picas) on
both left and right-hand margins.  Use 10~point type, with a vertical
spacing of 11~points.  {\bf Abstract} must be centered, bold, and in
point size 12. Two line spaces precede the Abstract. The Abstract must
be limited to one paragraph.
\end{abstract} 
 
\section{GENERAL FORMATTING INSTRUCTIONS} 
 
Papers are in 2 columns with the overall line width of 6.75~inches
(41~picas).  Each column is 3.25~inches wide (19.5~picas).  The space
between the columns is .25~inches wide (1.5~picas).  The left margin
is 1~inch (6~picas).  Use 10~point type with a vertical spacing of
11~points.  Times Roman is the preferred typeface throughout.
 
Paper title is 16~point, caps/lc, bold, centered between 2~horizontal
rules.  Top rule is 4~points thick and bottom rule is 1~point thick.
Allow 1/4~inch space above and below title to rules.
 
Reviewing is double-blind, so do not include author names, affiliations, or any
other identifying information in the original submission.  If you include urls
to supplementary material, make sure the urls also do not disclose your identity.

After a paper is accepted, for the camera-ready submission, Authors' names are
centered, initial caps.  The lead author's name is to be listed first
(left-most), and the Co-authors' names (if different address) are set to
follow.  If only one co-author, center both the author and co-author,
side-by-side.
 
One-half line space between paragraphs, with no indent. 
 
\section{FIRST LEVEL HEADINGS} 
 
First level headings are all caps, flush left, bold and in point size
12. One line space before the first level heading and 1/2~line space
after the first level heading.
 
\subsection{SECOND LEVEL HEADING} 
 
Second level headings must be flush left, all caps, bold and in point
size 10. One line space before the second level heading and 1/2~line
space after the second level heading.
 
\subsubsection{Third Level Heading} 
 
Third level headings must be flush left, initial caps, bold, and in
point size 10.  One line space before the third level heading and
1/2~line space after the third level heading.
 
\vskip .5pc 
Fourth Level Heading 
 
Fourth level headings must be flush left and initial caps.
One line space before the fourth level heading and 1/2~line space
after the fourth level heading.
 
\subsection{CITATIONS, FIGURES, REFERENCES} 

 
\subsubsection{Citations in Text} 
 
Citations within the text should include the author's last name and
year, e.g., (Cheesman, 1985). Reference style should follow the style
that you are used to using, as long as the citation style is
consistent.

For the original submission, take care not to reveal the authors' identity through
the manner in which one's own previous work is cited.  For example, writing
``In (Bovik, 1970), we studied the problem of AI'' would be inappropriate, as
it reveals the author's identity.  Instead, write ``(Bovik, 1970) studied the
problem of AI.'' 
 
\subsubsection{Footnotes} 
 
Indicate footnotes with a number\footnote{Sample of the first
footnote} in the text. Use 8 point type for footnotes.  Place the
footnotes at the bottom of the page on which they appear.  Precede the
footnote with a 0.5 point horizontal rule 1~inch (6~picas)
long.\footnote{Sample of the second footnote}
 
\subsubsection{Figures}  
 
All artwork must be centered, neat, clean, and legible. Figure number
and caption always appear below the figure.  Leave 2 line spaces
between the figure and the caption. The figure caption is initial caps
and each figure numbered consecutively.
 
Make sure that the figure caption does not get separated from the
figure. Leave extra white space at the bottom of the page rather than
splitting the figure and figure caption.
\begin{figure}[h] 
\vspace{1in} 
\caption{Sample Figure Caption} 
\end{figure} 
 
\subsubsection{Tables} 
 
All tables must be centered, neat, clean, and legible. Table number
and title always appear above the table.  See
Table~\ref{sample-table}.
 
One line space before the table title, one line space after the table
title, and one line space after the table. The table title must be
initial caps and each table numbered consecutively.
 
\begin{table}[h] 
\caption{Sample Table Title} 
\label{sample-table} 
\begin{center} 
\begin{tabular}{ll} 
\multicolumn{1}{c}{\bf PART}  &\multicolumn{1}{c}{\bf DESCRIPTION} \\ 
\hline \\ 
Dendrite         &Input terminal \\ 
Axon             &Output terminal \\ 
Soma             &Cell body (contains cell nucleus) \\ 
\end{tabular} 
\end{center} 
\end{table} 
 
\newpage 

\subsubsection*{Acknowledgements} 
 
Use unnumbered third level headings for the acknowledgements title.
All acknowledgements go at the end of the paper.
 
 
\subsubsection*{References} 
 
References follow the acknowledgements.  Use unnumbered third level
heading for the references title.  Any choice of citation style is
acceptable as long as you are consistent.
 
 
J.~Alspector, B.~Gupta, and R.~B.~Allen  (1989). Performance of a 
stochastic learning microchip.  In D. S. Touretzky (ed.), {\it Advances 
in Neural Information Processing Systems 1}, 748-760.  San Mateo, Calif.: 
Morgan Kaufmann. 
 
F.~Rosenblatt (1962). {\it Principles of Neurodynamics.} Washington, 
D.C.: Spartan Books. 
 
G.~Tesauro (1989). Neurogammon wins computer Olympiad.  {\it Neural 
Computation} {\bf 1}(3):321-323. 
 
\end{document} 



