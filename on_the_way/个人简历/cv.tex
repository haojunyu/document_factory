\documentclass[11pt,a4paper,nolmodern]{moderncv}
\usepackage[noindent,UTF8]{ctex}
\usepackage{info}

\title{同济大学电子与信息学院}
\myquote{自强不息,厚德载物。}{}

\begin{document}
\setmainfont{Adobe Fangsong Std}
\setsansfont{Adobe Kaiti Std}

%\hyphenpenalty=10000
\maketitle

\section{求职意向}
\cvhobby{应聘职位}{算法工程师}
\cvhobby{兴趣方向}{数据分析,机器学习}
\cvhobby{应聘理由}{本科阶段,参加了一些竞赛并开发过几个系统.不过在读研阶段,伴随大数据时代的来临,想要在数据挖掘和机器学习方面做点研究,所以在这个阶段除了自学了coursera上Andrew\,Ng的Machine\,Learning课程,还在导师的指导下对deep\,learning和bayesian network作了些研究,完成了一篇论文.希望应聘这个职位,能够在实践中应用我的所学.}

\section{教育背景}
\tlcventry{2013}{0}{同济大学,电信院,研究生在读,2016年毕业}{}{}{}{
\begin{itemize}
  \item 2014年,完成论文<DISCRIMINATION OF ADHD CHILDREN BASED ON DEEP BAYESIAN NETWORK> 
\end{itemize}}
\tlcventry{2009}{2013}{苏州大学,计算机科学与技术学院,本科,2013年毕业}{}{}{}{
\begin{itemize}
  %\item 2010年,获苏州大学学习优秀奖学金
  \item 2011年,获高教社杯全国大学生数学建模竞赛本科组全国二等奖
  %\item 2011年,获“国信蓝点杯”全国软件专业人才设计与开发大赛江苏赛区C语言程序设计本科组二等奖
  \item 2011年,获苏州大学人民综合一等奖学金、苏州大学三好学生、国家励志奖学金
  %\item 2012年,获美国大学生数学建模竞赛二等奖
  \item 2012年,获“蓝桥杯”全国软件专业人才设计与创新大赛Java	本科组江苏赛区一等奖,全国总决赛三等奖
\end{itemize}}
\section{能力}
\subsection{开发}
\cvcomputer{语言}{C/C++(熟悉), C\#, Java}
           {Web}{HTML, ASP .Net}  
\cvcomputer{数据库}{(熟悉)MySQL, SQL Server}
			{工具}{Matlab, Shell编程, Python}

\subsection{其他}
\cvcomputer{项目管理}{ github }
           {文字编辑}{\LaTeX{},VIM}

\section{经历}
\subsection{项目经历}
\tlcventry{2014}{2015}{基于深度贝叶斯网络的多动症判别分析}{}{}{}{
\begin{itemize}
 \item 该项目是我小论文研究的内容,其应用深度贝叶斯网络对小孩的FMRI图像进行多动症的判别分析.
 \item 该研究涉及深度学习和贝叶斯网络以及SVM分类.
 \item 该研究让我将机器学习的方法应用于医学实践,使我对未来的数据分析更加的看好.
\end{itemize}}
\vspace{0.5em}

\tlcventry{2014}{2015}{基于 Django 的web应用开发}{}{}{}{
\begin{itemize}
 \item 该web应用是基于python的Django框架,旨在搭建Nuance公司的训练管理平台.
 \item 该应用接受员工上传的训练数据集,让服务器进行训练和显示.
 \item 该应用由python编写,涉及mysql,html等技术.
\end{itemize}}
\vspace{0.5em}

\tlcventry{2014}{2015}{基于 ENLU 的 android 应用开发}{}{}{}{
\begin{itemize}
 \item 该android应用是基于Nuance开发的ENLU库, 旨在帮用户语音管理手机应用.
 \item 该应用通过语音输入, 分析用户的意图和对象, 并完成相应操作.
 \item 在Nuance公司实习的期间,让我对自然语言理解有了一定的了解.
\end{itemize}}
\vspace{0.5em}

\tlcventry{2012}{2013}{虚拟桌面系统调度服务器软件的设计和实现}{}{}{}{
\begin{itemize}
 \item 虚拟桌面系统分为调度服务器,应用服务器和客户端,旨在为云计算提供支持.
 \item 本人负责调度服务器的开发.主要由任务调度,负载均衡和用户管理三个模块构成.
 \item 该软件用c语言编写,采用Socket通信和多线程编程,并运行在Linux环境下.\\该软件目前只是用于实验室试验阶段。
\end{itemize}}
\vspace{0.5em}

\tlcventry{2010}{2011}{研究生管理系统}{}{}{}{
\begin{itemize}
 \item 该系统用来帮助导师管理研究生的日常工作和学习情况,目前依旧在苏大运行着.
 \item 采用SOA架构和Web Service技术在.NET平台下开发的一个基于B/S模式的管理系统
 \item 该管理系统的实现加深了我对数据库的应用和理解。
\end{itemize}}


\subsection{社会工作}
\tlcventry{2015}{0}{SAP公司开发运维部门}{}{}{}{
\begin{itemize}
 \item ELK系统搭建
 \item 用户行为以及系统状态分析。
\end{itemize}}

\tlcventry{2014}{2015}{Nuance公司研发部}{}{}{}{
\begin{itemize}
 \item 将ENLU核心库应用到android的app上。
 \item 基于Django开发任务管理系统。
\end{itemize}}

\tlcventry{2013}{2014}{同济大学电子与信息学院研究生会学习部部长}{}{}{}{
\begin{itemize}
 \item 负责策划与组织管理文化节学术讲座、学术沙龙系列活动 
 \item 参与策划和组织“同济女生节活动”
\end{itemize}}



\section{自我评价}
\cvhobby{学习}{本科阶段参加了不少竞赛并获得了些成绩。其次还做过一个管理系统。而硕士阶段主要精力投入在机器学习方面,熟悉主流的Machine Learning算法,自学能力强,善于总结,喜欢写\href{http://haojunyu.gitcafe.io}{博客}记录自己的感悟。}
\cvhobby{生活}{喜欢旅游,随手拍照记录日常点滴。此外还喜欢玩游戏和跑步,以此来释压。}

\end{document}

