\documentclass[10pt,a4paper]{article}
\usepackage{amsmath}
\usepackage{amsfonts}
\usepackage{amssymb}
\usepackage{CJK}													% 支持中文
\usepackage{indentfirst}                 						% 首行缩进宏包
\usepackage{graphicx}											% 支持图片的插入
\usepackage{subfigure}											% 支持插入子图
\usepackage[colorlinks,citecolor = blue, linkcolor=blue,hyperindex,CJKbookmarks]{hyperref}	% 链接功能


\graphicspath{{figs/}}                              				% 图片文件夹路径
\setlength{\parindent}{2em}										% 缩进距离为2个字符位置
\newcommand{\song}{\CJKfamily{song}}     						% 宋体
\newcommand{\hei}{\CJKfamily{hei}}       						% 黑体
\newcommand{\fs}{\CJKfamily{fang}}         						% 仿宋
\newcommand{\kai}{\CJKfamily{kai}}       						% 楷体
\newcommand{\li}{\CJKfamily{li}}         						% 隶书
\newcommand{\you}{\CJKfamily{you}}       						% 幼圆

\begin{document}

\begin{CJK*}{UTF8}{gkai}
%============================++题目和作者++================================
\title{揭开桌面虚拟化的神秘面纱}					   					% 题目
\author{TOMISLAV PETROVI\'{C} 		\\  
        Vetropack Stra\u{z} d.d. Hum na Sutli\\    
        Hum na Sutli 203, 49231 Hum na Sutli \\
        CROATIA \\
		tomislav.petrovic@vetropack.hr http://www.vetropack.hr  \\                                   
  \and                                  %如有多作者, 用\and 
        KRE\u{S}IMIR FERTALJ \\
        University of Zagreb\\
        Faculty of Electrical Engineering and Computing, Unska 3\\
		CROATIA \\
		kresimir.fertalj@fer.hr http://www.fer.hr\\
		}
%============================++++++++++++=================================
\date{}                                             				% 显示作者,不显示日期
\maketitle                                          				% 生成标题
\tableofcontents 												% 生成目录
\clearpage


%============================++摘要和关键字++===============================
\newcommand{\cnabstract}
{
本文介绍了虚拟化技术的概念。无论是服务器虚拟化还是桌面虚拟化都作了简要的介绍,但桌面虚拟化讨论了更多的细节。由于新的应用程序使得计算机系统日益庞大,同时新的客户端和市场的要求,采取一些措施以缓解计算机的管理和有效地保护数据势在必行。此外,崩盘的恢复技术也得提升。虚拟化技术已经解决了许多这些要求,并越来越受欢迎,因此,许多公司决定构建虚拟服务器,虚拟存储和虚拟桌面。本文罗列了实施虚拟桌面和服务器的原因以及这样做的主要好处。对于虚拟环境的一些缺点,也作了介绍,解释和与优势的比较。
}
\newcommand{\cnkeywords}{桌面虚拟化,虚拟机,客户端基础设施,连接管理器,瘦客户机}
%============================++++++++++++=================================
\begin{center}
\begin{minipage}[c]{12cm}										% 小页环境-摘要
\hei 摘要:\kai \cnabstract\\
\hei 关键字:\kai \cnkeywords\\
\end{minipage}
\end{center}
\clearpage



%================================主体====================================
\section{引言}
许多不同的计算机上的人工管理是艰难,繁琐和费时的工作。用户在本地计算机上对软件和硬件的安装和重新按照的选择存在不同。为了避免这种情况,尽管应用程序的安全性由服务器维护,但是终端服务器被广泛地用于用户可以使用和访问的单独位置。桌面虚拟化的概念跟它很相似。 


桌面虚拟化的思想是,以用一个虚拟的桌面来取代用户的计算机。尽管用户的计算机位于桌子上,但是虚拟计算机处于数据中心的服务房间的保护中。用户用桌面客户端连接到他们的虚拟机上作为一个最小的资源。桌面客户端可以是没有任何预先安装的软件,就像操作系统等。此外,桌面客户端可以是有少量内存和处理速度较慢的典型旧PC;它可以是MAC PC或者一些特殊的设备比如Pano或Wyse。桌面客户端仅用于连接到虚拟机,比如使用终端服务器,但它为用户提供了更大的灵活性和自主性。 


这种集中的主要优点是低成本和易于管理。有一些优点已经被虚拟服务器包含,虚拟计算机也能达到,像高可靠性和系统的可用性,更新的灵活,移动访问等。本文介绍了虚拟化的概念,桌面虚拟化在领头公司的使用,讨论它用法的原因并给出主要的优势,最后谈一下虚拟化桌面不完善的地方。


\section{虚拟化}
在计算机科学中,虚拟化这个词用于对计算机资源的抽象化,如内存,网络,处理器,应用等\cite{2},\cite{5},\cite{6}。当我们谈论虚拟化时,我们是指平台的虚拟化,这实际上表示计算机的虚拟化或操作系统的虚拟化\cite{2}。虚拟化的核心被称为虚拟计算机管理程序,它是用来处理虚拟计算机的。虚拟化可以根据虚拟化和管理的操作模式水平来分类。根据操作的模式,我们区分非托管的和托管的虚拟机管理程序。而第一级直接操作硬件(如Xen,ESX Server),第二级在另一个作业系统上执行(如Microsoft Virtual PC和VMware Workstation)\cite{3},\cite{12},\cite{16}。基于虚拟化程度的分类是比较复杂的。主要的分类如下:
\begin{itemize}
\item[a)] 完全虚拟化 - 实现完整的平台,并能够使用虚拟操作系统。客户机操作系统完全是由物理设备抽象的,它被虚拟化后没啥影响。完全虚拟化的典型有:Qemu,Vmware,Vmware Workstation,Virtual Box等。
\item[b)] 硬件辅助虚拟化 - 虚拟化部分是来自硬件。在操作系统中有指令直接访问硬件,无需任何操作系统的修改或仿真。例子有:IBM POWER,Intel VT和 AMD V指令。在Linux中有KVM作为辅助虚拟系统的最高级的硬件。
\item[c)] 部分虚拟化只实现了一些具体部分的抽象。这种解决方案将允许资源的划分和过程的隔离,但不是另一个操作系统的开始。对于每一个进程和用户要使用单独的地址空间。
\item[d)] Para虚拟化能使计算机运行独立的操作系统。这些操作系统必须为hyper calls主机作些变更。Para虚拟化要用能够直接和虚拟层直接沟通的hyper calls替换未虚拟化的指令来修改操作系统的内核。一个例子是Xen。因为平台自身和内核的独立性,Para 虚拟系统在和不同系统组合上以及从一个平台迁移到另一平台上有些限制,但是因为他们在工作量上的独立性,所以相较于部分虚拟化和完全虚拟化他们有许多更好的性能。
\item[e)] 没有硬件虚拟化的操作系统虚拟化,这就意味着相同操作系统的不同实例能够运行在相同的硬件上。这种虚拟化的例子是Open Vz和Linux Vserver。
\end{itemize} 


如今虚拟化的形式有很多。 服务器基础设施的虚拟化是最主流的形式,而此时最新的一个是桌面虚拟化。这些方式也被使用:服务器虚拟化,存储虚拟化,网络虚拟化,应用虚拟化,桌面虚拟化\cite{10},\cite{12},\cite{13},\cite{16}。 


如今的使用虚拟化的主要原因是它可以以不同的方式被使用。其中之一是服务器整合,因此可以节省服务器空间。虚拟化已经在操作系统的开发人员和内核之间找到了自己的用户。用于调试的设施也很不错。


最终用户将为测试新的应用,新的操作系统等来使用虚拟化,新的操作系统等。其中一个将虚拟化充分利用的最有吸引力的方法是用户桌面的虚拟化。本文将进一步介绍桌面虚拟化以及它主要的特点。


\section{桌面虚拟化}
术语虚拟桌面描述了两种互补的技术:基于服务器和客户端托管的虚拟化。在二者间,一个标准的桌面操作系统被封装在能够被用户访问的虚拟机中。当谈及客户端托管的虚拟化,虚拟机是位于客户端并在客户端操作的。客户端运行操作系统以及用于虚拟化的应用程序。基于服务的虚拟桌面在服务器上运行多个虚拟机,而用户在他/她的PC,客户端或者特殊设备比如Pano或者Wyse上获得远程的显示。


哪些人会使用这些技术,依赖于所必须满足的要求。例如,如果一个人需要一个解决公司数据安全的方案,基于服务的解决方案是一个不错的选择。另一方面,如果一个人需要为正在旅行的员工或者需要脱机工作或者一个有限带宽的网络连接的人提供公司的台式机和一组应用,那么客户端托管是正确的选择。


\subsection{了解虚拟桌面基础设施}
虚拟桌面基础架构(VDI)是一个新的桌面虚拟化,它允许客户端操作系统运行基于服务的虚拟 
机。客户端桌面可以运行在瘦客户端和台式电脑上。存储,执行和虚拟桌面的管理是放置在数据中心。一个典型的虚拟桌面基础设施的实现有下面的组件:
\begin{itemize}
\item[a)] 系统管理程序 – 使机器虚拟化和主管主机镜像的软件。正如我们前面提及的,有两种类型的管理程序。第一个是软件系统,它直接运行主机的硬件上。操作系统上运行比虚拟化管理程序更高的一层。通常情况下,对于这种类型的虚拟化我们可以发现像裸金属和为这种类型的虚拟化的本机的名称。第二种类型是在操作系统中运行的软件。客户机操作系统运行在硬件之上的第三个层次。
\item[b)] 虚拟机管理器 – 应用程序的目的是管理虚拟机。用户可以创建,启动和停止虚拟机。用户还可以查看和控制每个虚拟机控制台,预览性能数据和一些其他东西的状态。
\item[c)] 连接代理 – 软件安装在独立的虚拟服务器上。在同一个网络中的服务器可以有一个或多个。连接代理管理来自瘦客户端或者软件客户端的连接请求,并将这些请求连接到虚拟桌面的连接池中。通常情况下,术语“连接管理器”或“虚拟桌面管理器”被用于连接代理。
\item[d)] 虚拟机 - 一个真正的在执行程序的计算机的软件实现。这个虚拟机独立于任何硬件。现在有两种类型的虚拟机:流程虚拟机和系统虚拟机。流程虚拟机被设计为只运行单一的程序,这和运行着操作系统的系统虚拟机相反。对于系统虚拟机我们也可以查看术语“硬件虚拟机”。
\end{itemize} 

\subsection{VDI - 实现}
有两个VDI的核心架构 – 静态和动态。选择哪种架构依赖于公司的需求。如果想要为每个用户提供一个独立的虚拟机,静态模式可能是个好选择。在静态(持续性)的架构选项中,用户和虚拟机直接存在存在一对一的映射关系。每个用户都有他或她的虚拟机,因此,用户越多,虚拟机也将被创建和管理的越多。虚拟机通常被存储在一个存储区域网络(SAN)或网络附加存储(NAS)。虚拟桌面通过VDI呈现给标准台式PC或瘦客户端。在动态架构选项(非持久性​​),单主虚拟机镜像位于系统管理程序中,VDI系统将会自动按照用户的需求来复制这个镜像。用户的应用程序是根据用户配置文件和配置权限分配的,并且用户数据是通过文件夹重定向集中存储在服务器上的。只有单一的虚拟机镜像要维护,管理费用和支持成本显著降低,而动态架构的桌面环境的配置是有需求的。 
\begin{figure}[!htbp]
	\centering
	\caption{虚拟桌面基础设施(VDI)}
    	\includegraphics[scale=0.7]{figs/pic1.pdf}
    \label{fig:subfig4}
\end{figure}

\subsection{VDI的优点}
虚拟化带来了许多优点,本文在这一部分介绍了一些。
\begin{description}
\item[虚拟化使数据更安全] 人们可以防止用户从虚拟机中复制文件或从外部设备锁定镜像。敏感数据 
存储在数据中心的服务器上,而不是在远程设备上像笔记本电脑。在这种情况下,如果一个设备被盗, 
信息将会得到保护。 
\item[简化了IT的管​​理]  存在用于管理和控制的优秀管理工具。管理数百或多个物理设备是迫切需要的以及高效率的。用虚拟化技术,虚拟桌面和软件的部署是非常简单和快捷的。此外,管理员可以远程启动,停止和控制虚拟机。 
\item[方便管理VDI系统]  基于虚拟机的模板,我们可以确保用户被迫遵守有关法律法规和公司章程规定。
使用每个价值40-50美元的工作场所的VDI部署将没有硬件升级的必要。当然,这取决于所购买的产品和公司的要求。人们可以选择免费软件,使用旧的物理PC作为桌面客户端,也可以使用价值1000美元的工作场地,里面有想Pano和Wyse等这样的设备。。即便如此,基于虚拟工作场所的维护成本仍然是较低的。 
\item[可以简单的复制或移动的镜像]  利用VDI技术和快照技术,可能将虚拟机回退到台式机的任何不同的状态。这给最终用户带来了很大的灵活性,管理员可以快速的恢复数据或整个虚拟机。
\item[减少了碳在我们这个星球上的足迹\cite{14}]  瘦客户端比一台台式电脑消耗更少的电力。瘦客户端在正常工作模式下比物理计算机“待命”模式使用更少的电量。 
\item[使用任何设备连接到虚拟桌面上]  如果有DSL,那么对连接就没有限制。而它所需的所有仅是一个VPN帐户和一个配置过的和分配过的虚拟桌面。 
\item[可以有多个(不同的)虚拟机(台式机)运行在一台物理服务器上] 通过整合服务器,我们最大限度地利用资源;虚拟桌面也是如此。 
\item[集成度高度简化]  它几乎可以与所有的硬件集成,是一个只有300兆赫处理器和128 MB的RAM或更弱。没有必要为了实现VDI技术而购买更昂贵的硬件 。 
\end{description}


\subsection{优点与缺点}
在这一部分,我们比较了VDI的优点和它的缺点。出现的大多数问题是因为较差的兼容性。由于运行并管理虚拟机的应用程序,所以虚拟机在性能方面有些慢,剥离硬件意味着要通过我们选择的虚拟解决方案访问设备,并且这可能不是总是行得通。像质量差的视频和音频支持还有像PCMCIA外围设备支持这样的问题是频繁遇到的。许多图形或处理器密集型的桌面应用工作在物理基础设备上比在虚拟基础设备上要好。当所有桌面共享主机服务器的处理器和内存时,问题就会出现了。至于图像和音频,所有的执行任务是运行在服务器上,并通过RDP呈现给用户。这是众所周知RDP对流不进行优化。 


\begin{table}[!hbp]%开始表格
\begin{tabular}{c c}%开始绘制表格
\hline %绘制一条水平的线
\textbf{Advantages} 			& 			\textbf{Disadvantages}				\\
\hline
Utilization increased  		&  			Performances degraded 				\\
Live Backup and migrations  	&			No real standards					\\
Cloning and snapshots  		&			storage								\\
Energy saving   				&			Complex root cause analysis			\\
Administration is easier  	&			New concepts and tools				\\
-  							&			Low graphics							\\
Go green						&			Multimedia performance - poor		\\
Manage and monitor regulatory compliance		&		-						\\
Flexibility and mobility		&			Peripheral support - poor			\\
Reduced costs				&			Cannot work offline					\\
Disaster recovery			&			Single point of failure (SPOF)		\\
Data security				&			can’t virtualize just everything		\\
Lower noise					&												\\
\hline
\end{tabular}
\caption{TABLE 1} %表格的名称
\end{table}


虚拟桌面的另一个缺点是脱机工作是不可能的。当出现网络问题时,用户的工作就被中断了,并且他们得等到网络能够再次可以通畅为止。VDI不能在隧道或在飞机上使用。 


此外,如果一个人在一个昂贵的SAN或NAS上装有许多虚拟机试图通过使用虚拟化桌面来降低成本时,存储问题将会是一个根源性的问题;需要找到价廉物美的存储器。 


主机上的其他问题可能会导致所有的虚拟机停止工作,但这个问题通过冗余装载部分地解决了。在虚拟化方面没有真正的标准。虚拟化意味着对整个系统进行大的改变,肯定会造成新的问题。对这个新问题,它可以非常复杂的提出一种解决方案,并且它可能需要新的工具以及知道如何在IT中心实现。 


此外,还有一个问题是与旧的应用程序相关的,它们依赖于一些硬件组件的MAC地址,还有就是使用COM+的应用,以及安装系统驱动或启动时的服务的应用。当然,这列问题还远远没有完成。但这并不是意味着它不可能用上述指定的功能来虚拟化一些应用程序。但可以预计在这方面的问题。


\subsection{VDI - 趋势和统计}
在2008年进行的一项调查结果显示(IDG - 国际数据组别),虚拟化增长很迅速\cite{7},\cite{9}。 大部分原因是在降低成本。此外,因为更加安全和可管理的桌面环境使人们对桌面电脑的兴趣在增加。桌面环境采用集中管理和近几年才创建的配置功能。一般情况下,桌面虚拟化是有利的。 IDG的调查显示,有超过50%的公司部署过VDI,结果显示桌面虚拟化满足他们的期望。 


桌面虚拟化的好处对于这个问题,54%的受访者回答说降低成本,54\%回答说更易于管理的桌面环境,52\%回答说有能力在集中的地方提供电脑和其他有软件的客户端设备。 


在VDI上正常的投资是每个桌面上600美元到900美元,而桌面客户端的成本约300美元。一个可以经营很好,但大部分这样就不是很好了。在虚拟机管理程序上的投资也是必不可少的。虚拟机管理程序(如VMware ESX, 微软的Hyper-V,思杰XenServer和其他)不同于连接代理\cite{1},\cite{2},\cite{4},\cite{6}。许多人使用微软的RDP(远程数据协议)来处理客户端与服务器的连接,并且其他使用专门的协议进行压缩和其他的优化技术\cite{2}。 Qumranet有一个渲染的协议,称为独立计算环境的简单协议。这个协议旨在支持多媒体。思杰ICA - 独立计算架构通信协议,并有自己的方法来优化。 


今天,许多解决方案可以在市场上发现,但是直到现在,还没有发现可以完全代替经典物理基础设施。
\begin{figure}[!htbp]
	\centering
	\caption{虚拟化的趋势}
    	\includegraphics[scale=0.4]{figs/pic2.pdf}
    \label{fig:subfig4}
\end{figure}


\section{结论}
桌面虚拟化需要运行在数据中心服务来控制虚拟桌面的虚拟机管理程序。直到现在为止,没有真正的标准建立。在一个公司里,超过一个以上的虚拟机管理程序可以建立,但对于一些工作最好还是使用传统的台式电脑。到目前为止,在多于一个不同的虚拟机管理程序,超过一个管理软件是必须的。在最后几个月中,能够同时管理更多的虚拟机管理程序和物理机器的软件将会在市场上出现。 


没有通用的秘诀,这在很大程度上取决于一个人的需求。如果一个人需要安全性和灵活性,也能更好地运作并始终保持对网络的连接,他/她会选择一个供应商像提供类似MokaFive的Live PC。有了它们,人们可以使用最好的基于服务器和客户端托管的桌面虚拟化技术。 


本地虚拟化对移动工作人员是有意义的,他们能够使用单独的操作系统,其中的一个或多个将会被用于商业用途,还有个人用途。 


桌面虚拟化技术提供了许多优点,但要注意,为了紧跟潮流,为了降低成本,为了使管理更容易,为了实现用户的灵活性或别的东西。必须注意什么是主要目标和采用什么样的技术来实现。


\bibliographystyle{unsrt}										% 按引用的先后顺序排列,比较次序为作者,年度和标题
\bibliography{mybib}												% 引用文件数据库在bib.bib文件中

\clearpage     
\end{CJK*}

\end{document}