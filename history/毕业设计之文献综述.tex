\documentclass[10pt,a4paper]{article}	
\usepackage{amsmath}
\usepackage{amsfonts}
\usepackage{amssymb}
\usepackage{cite}												% 支持引用多篇文献
\usepackage{CJK}													% 支持中文
\usepackage{indentfirst}                 						% 首行缩进宏包
\usepackage{graphicx}											% 支持图片的插入
\usepackage{subfigure}											% 支持插入子图
\usepackage[colorlinks,citecolor = blue, linkcolor=blue,hyperindex,CJKbookmarks]{hyperref}	% 链接功能
\usepackage{fancyhdr}											% 添加页眉

\graphicspath{{figs/}}                              				% 图片文件夹路径
\setlength{\parindent}{2em}										% 缩进距离为2个字符位置
\newcommand{\song}{\CJKfamily{song}}     						% 宋体
\newcommand{\hei}{\CJKfamily{hei}}       						% 黑体
\newcommand{\fs}{\CJKfamily{fang}}         						% 仿宋
\newcommand{\kai}{\CJKfamily{kai}}       						% 楷体
\newcommand{\li}{\CJKfamily{li}}         						% 隶书
\newcommand{\you}{\CJKfamily{you}}       						% 幼圆


\begin{document}

\begin{CJK*}{UTF8}{gkai}
%============================++题目和作者++================================
\pagestyle{fancy} 
\lhead{}
\chead{\small{苏州大学本科生毕业设计(文献综述)}}
\rhead{}
\title{桌面虚拟化技术研究综述}					   					% 题目
\author{郝俊禹\thanks{Email:haojunyu2012@gmail.com}}				% 作者
%============================++++++++++++=================================
\date{}                                             				% 显示作者,不显示日期
\maketitle                                          				% 生成标题
\tableofcontents 												% 生成目录
\clearpage


\section{引言}
云计算是一种利用互联网实现按需、便捷地访问共享资源池的计算模式;是分布式、资源虚拟化与数据中心管理技术、互联网技术的融合、可共享的资源包括计算设施、存储设备、应用程序等\cite{1}。云计算具有成本低、可用性高、扩展性强、按需服务等应用优势。


云计算中最关键的技术是虚拟化技术,虚拟化技术是把一个物理单元虚拟成多个逻辑单元,供多个应用一起使用\cite{2}.虚拟化技术打破了国有的物理硬件之间的隔离,将网络中所有的设备统一管理,形成共享资源池,通过虚拟机等监视程序将这些资源当成一个实体来管理,根据工作负载的大小动态地分配硬件资源,从而实现资源共享、资源定制和按需服务。


\section{桌面虚拟化的相关概念及历史}
\subsection{桌面虚拟化的定义}
桌面虚拟化,最简单的定义是将桌面或者客户端操作系统与原来的物理硬件进行分割,实现更灵活的使用\cite{3}。维基百科给出的定义是:“虚拟桌面(VDI)是一种基于服务器的计算模型,并且借用了传统的瘦客户端的模型,但是让管理员与用户能够同时获得两种方式的优点:将所有桌面虚拟机在数据中心进行托管并统一管理;同时用户能够获得完整PC的使用体验。”


\subsection{桌面虚拟化的发展}
\subsubsection{桌面虚拟化的雏形}
一种是远程桌面,远程桌面是微软在WINDOWS 2000 SERVER以及XP等操作系统内置的功能组件。当某台计算机开启了远程桌面连接功能后,我们可以在网络的另一端通过远程桌面功能控制这台计算机。


另一种则是桌面操作系统的虚拟化。这种应用模式是被人们第一次以桌面虚拟化技术接受的概念。虚拟化技术刚起步的时候,一些厂商将此定义为桌面虚拟化技术。这个就阶段的桌面虚拟技术都是基于第三方软件的,用于PC上的桌面系统的虚拟化解决方案,本身解决的仍然是操作系统的安装环境与运行环境的分离,不依赖特定的硬件\cite{4}。


\subsubsection{桌面虚拟化技术}
第一代桌面虚拟化将用户操作系统环境与系统实际运行环境拆分并将远程桌面的远程访问能力与虚拟操作系统结合了起来,同时用户端的虚拟桌面实现集中化,由服务器端进行集中化的管理,用户通过网络借助物理机器(PC、瘦终端)访问属于个人的桌面。


第二代桌面虚拟化技术从管理角度,进一步将桌面系统的运行环境与安装环境拆分,从而大大降低了管理复杂度与成本,实现桌面虚拟化的简化与可用化,提高管理效率。


\section{虚拟化的应用实现}
\subsection{桌面虚拟化的解决方案}
桌面虚拟化技术是集成了服务器虚拟化、客户机、远程协议等多种技术,一个完整的桌面虚拟化解决方案一般主要由4部分组成:服务器及虚拟化软件、虚拟化的桌面主机、连接代理和客户机\cite{5}。
\begin{description}
\item[服务器及虚拟化软件]	通过虚拟化软件在服务器上实现对桌面操作系统的虚拟化和运行管理,具有数据备份、资源分配、应用部署等多种辅助功能。
\item[虚拟化的桌面主机]		由服务器虚拟化软件提供的虚拟机环境,用来安装各种桌面操作系统和应用软件。用户可以通过远程连接协议来连接虚拟化的桌面主机,在虚拟化的桌面主机上进行各类操作与使用。
\item[连接代理]	它是桌面虚拟化技术的关键,代理决定了用户可以访问或者连接哪个虚拟化的桌面。连接代理是一种成熟的管理软件产品,通过连接代理可以实现提供远程主机连接和自动的部署。
\item[客户机]		需要运行虚拟桌面的工作站。工作站可以是瘦客户端机、PC机和笔记本。瘦客户机一般是由嵌入式芯片和操作系统构成,它的处理能力有限。为提高桌面虚拟化应用的交互体验,PC机和笔记本可以通过使内置的连接代理登录并运行虚拟桌面。
\end{description}


\subsection{桌面虚拟解决方案的主要代表}
当前在桌面虚拟化领域,主要的桌面虚拟化方案有微软(Microsoft)的桌面虚拟化技MED-v、思杰(Citrix)的XenDesktop桌面虚拟化方案、广州方景的信息桌面虚拟化解决方案(幻影桌面)和VMware公司的VMware View桌面虚拟化方案\cite{6}。
\begin{description}
\item[思杰(Citrix)]	采用VDI架构,能够为虚拟机分配独立的资源,虚拟机支持Winxp、Win7、Linux等操作系统,而在后端,使用单一的镜像技术批量动态生成虚拟机,使得整个桌面逻辑组成部分实现了完整的分离,并采用其独立的高效的ICA显示协议和对外设重定向等技术,能够得到更好的网络性能。它是目前市场上应用最广的桌面虚拟解决方案之一。
\item[微软(Microsoft)]	采用TS架构,虚拟机共享物理服务器资源,虚拟机支持Win Server2003、Win Server2008,利用了桌面的VPC产品创建和“下载”虚拟机镜像,而实际上提供给用户使用的是安装在虚拟机中的应用,并不是真正的虚拟桌面的解决方案。
\item[Phantosys(幻影桌面)]	是属于桌面虚拟化技术胖客户端的应用,并且是国产虚拟化平台的杰出代表。Phantosys桌面虚拟化平台是建立在以X86 PC为标准的IT基础架构之上的,专门针对企业大量分散的客户端集中管理的解决方案。
\item[Vmware View]	采用VDI架构,能够为虚拟机分配独立的资源,虚拟机支持微软、苹果、红旗等多种操作系统,其后台架构在Vmware Sphere上,虚拟机的远程访问使用了微软的RDP协议,实现了与Citrix类似的架构体系,并且在底层服务器虚拟化技术和桌面虚拟化方案的具体实施方面,同思杰相比具有独立的特点。同幻影桌面比较,它是瘦客户端,不再需要添加PC主机配置,并且系统部署简单,访问模型将更加灵活,能够提高安全性、降低运营成本和简化桌面管理。
\end{description}


\section{研究意义及挑战}
\subsection{研究意义}
文献\cite{5}采用VMware View4产品搭建虚拟化桌面环境,系统架构。实现 了桌面的集中管理和便捷使用,降低了实训机房的建设成本,方便对系统进行维护和还原。


文献\cite{7}采用VMware桌面虚拟化和云架构技术,建立基于云桌面的虚拟实训室建设方案,其具有云平台资源利用率高、集中化管理、节约成本等优点。而且对学校计算机设备更新和日常维护来说可以节省大量资金和人力。不过,因为存储上运行这全部的虚拟机文件,如果存储故障,会使系统宕机。所以备份非常重要。


文献\cite{8}采用Phantosys(幻影桌面)虚拟化平台进行教学辅助。为老师定制不同的教学环境,使学生能方便地选择他所要的学习内容。其对数字化校园、教育信息化的建设起重要推动作用。

\subsection{研究挑战}
目前,桌面虚拟话技术上依旧面临着很多问题:
\begin{enumerate}
\item	集中管理问题。多个系统整合在一台服务器中,在节省资源的同时,也面临着一个严重的问题,一旦服务器出现硬件故障,其上运行的多个系统都将停止运行。
\item	集中存储问题。默认情况下,用户的数据是保存在集中的服务器上的,每个虚拟桌面不知会占用多少的存储空间,这给服务器带来的存储压力非常巨大。
\item	虚拟化产品缺乏统一标准问题。由于各个软件厂商在桌面虚拟化技术的标准上尚未达成共识,至今尚无虚拟化格式标准出现,各虚拟化产品厂商的产品间无法互通。
\item	网络负载压力。局域网一般不会存在太大的问题,但是如果通过互联网就会出现很多技术难题,由于桌面虚拟化技术的实时性很强,如何降低这些传输压力,是很重要的一环。
\end{enumerate}
尽管桌面虚拟化面临的问题很多,但是人们对桌面虚拟化的前景是非常乐观的。随着人们对桌面虚拟化好处的认知的提高,以及对桌面虚拟化的需求的提出,相关技术的不断完善,桌面虚拟化必将普及。
		
		
\bibliographystyle{unsrt}										% 按引用的先后顺序排列,比较次序为作者,年度和标题
\bibliography{mybib}												% 引用文件数据库在bib.bib文件中

\clearpage     
\end{CJK*}
\end{document}
