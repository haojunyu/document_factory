\documentclass[10pt,a4paper]{article}	
\usepackage{amsmath}
\usepackage{amsfonts}
\usepackage{amssymb}
\usepackage{CJK}													% 支持中文
\usepackage{indentfirst}                 						% 首行缩进宏包
\usepackage{graphicx}											% 支持图片的插入
\usepackage{subfigure}											% 支持插入子图
\usepackage[colorlinks,citecolor = blue, linkcolor=blue,hyperindex,CJKbookmarks]{hyperref}	% 链接功能
\usepackage{listings}											% 支持代码显示
\usepackage{color}												% 给文字,表格和图形上色(68种)
\usepackage{xcolor}												% color包的扩展


%===========================c语言代码显示设置===============================
\lstset{
language=[ANSI]c,							% c语言
basicstyle=\small,							% 小号字体
numbers=left,								% 代码左边显行号
keywordstyle=\color{blue},					% 关键字用蓝色显示
numberstyle={\tiny\color{green}},			% 行号小小号,绿色	
numbersep=5pt,								% 行号与代码的距离
commentstyle=\small\color{red},				% 注释颜色和字号
backgroundcolor=\color[rgb]{0.95,1.0,1.0},	% 设置背景颜色
showspaces=false,							% 不显示空格
showtabs=false,								% 不显示\t
tabsize=4,									% \t的长短
frame=shadowbox, 							% 添加外框
framexleftmargin=5mm, 						% 外框左边的页边空白
rulesepcolor=\color{red!20!green!20!blue!20!},	%设置阴影颜色
breaklines=true,								% 自动断行
escapeinside=``,
extendedchars=false 							% 解决代码跨页时,章节标题,页眉等汉字不显示的问题
}
%=============================================================================

\graphicspath{{figs/}}                              				% 图片文件夹路径
\setlength{\parindent}{2em}										% 缩进距离为2个字符位置
\newcommand{\song}{\CJKfamily{song}}     						% 宋体
\newcommand{\hei}{\CJKfamily{hei}}       						% 黑体
\newcommand{\fs}{\CJKfamily{fang}}         						% 仿宋
\newcommand{\kai}{\CJKfamily{kai}}       						% 楷体
\newcommand{\li}{\CJKfamily{li}}         						% 隶书
\newcommand{\you}{\CJKfamily{you}}       						% 幼圆


\begin{document}

\begin{CJK*}{UTF8}{gkai}
%============================++题目和作者++================================
\title{可扩展的存储Java对象使用范围分区}					   			% 题目
%\author{郝俊禹\thanks{Email:haojunyu2012@gmail.com}}				% 作者
\author{Mariusz Bedla 		\\ 
        Kielce University of Technology\\    
        25-314 Kielce, al. 1000-lecia Pa ́ stwa Polskiego 7 \\
        m.bedla@tu.kielce.pl  \\                                   
  \and                                  %如有多作者, 用\and 
        Krzysztof Sapiecha \\
        Cracow University of Technology\\
        31-155 Kraków, ul. Warszawska 24\\
		pesapiec@cyf-kr.edu.pl\\
		}
%============================++++++++++++=================================
\date{}                                             				% 显示作者,不显示日期
\maketitle                                          				% 生成标题
\tableofcontents 												% 生成目录
\clearpage


%============================++摘要和关键字++===============================
\newcommand{\cnabstract}
{
可扩展的对象存储(SSO)应该允许存储和维护一个分布在许多节点的网络的巨大的数量的对象。RP*N是属于一个家庭的保序的结构,范围分区的可扩展的分布式数据结构(RP* SDDS)。SDDS的架构被设计来存储记录。对象和复杂对象之间的依赖关系的不同的结构的原因是需要一个新的体系结构RP*。本文介绍了一种新的面向对象版本的RP*N的体系结构和实现Java对象。此版本可用于在一个快速且可扩展的存储Java对象。在磁盘上存储作为主内存中的集合对象的序列化对象与绩效实施评价和比较。
}
\newcommand{\cnkeywords}{对象存储;可扩展性;SDDS;RP*;Java.}
%============================++++++++++++=================================
\begin{center}
\begin{minipage}[c]{12cm}										% 小页环境-摘要
\hei 摘要:\kai \cnabstract\\
\hei 关键字:\kai \cnkeywords\\
\end{minipage}
\end{center}
\clearpage


\section{引言}
对象存储(OS)是面向对象的数据库管理系统(OODBMS)\cite{1}的最重要的组成部分之一。 OS管理物理对象。这些都是记录并可能被视为字节的表。


OS应该允许存储和维护大量对象。因此OS需要强大且可扩展的计算机平台。多计算机可能是这样一个平台。一个多机适用于无共享并行架构\cite{2}这是非常大的数据库的可扩展性\cite{3}。在这样的结构的每一个计算机拥有本地存储器和磁盘,和作为服务器的数据的行为\cite{4}。所有的计算机都通过快速网络连接。计算机之间的通信是基于消息传递。


问题是如何分配的多机服务器之间的数据。有三种基本分区方案可能使用的\cite{4}:
\begin{itemize}
\item[-] 循环赛分区的数据是根据功能分区的ID模n,其中n是服务器的数量,
\item[-] 散列分区的数据分区根据一个散列函数,而
\item[-] 范围分区的分区数据,根据他们的范围。
\end{itemize}


SDDS是记录的文件,一个文件服务器上的多机扩展。SDDS文件的主要目的是存储数量庞大的记录,也使这些记录为众多客户快速和可靠的访问。记录都存储在所谓的水桶,定位于主存储器的服务器。每个水桶的容量是有限的。如果一个桶的负载达到某一临界水平,它进行分裂。一种新桶在另一个服务器上创建的,并且通常从分裂桶的一半的数据被移动到一个新的桶。


分布式面向对象的数据库管理系统使用碎片和分配计划。这些计划减少数据传输,没有本地数据的引用文件,不相关的数据访问和增加并发性\cite{5,6}。他们可能依赖于多种因素,例如一个类的结构的,频率和访问的一种类型,等等。相反的SDDS配售对象只依赖于一个关键的对象。


有许多架构的数据公布特殊标准文件,如RP*\cite{7},LH*\cite{8},DDH\cite{9}等。本文涉及的面向对象版本的RP* N架构存储Java对象序列化形式(OORP* N)的发展。作为分布式存储的Java对象的一部分,也可以使用这样的实施。


本文的组织如下。在下一节中,这项研究的动机。面向对象的一种体系结构的版本的RP*(OORP*)在3部分中介绍。在第4节,它的实现为Java对象。在第5节,实施对象序列化磁盘和主内存中的集合的比较。第6节总结研究。

\section{动机}
并非所有的适用于关系型数据库的架构的SDDS对面向对象的数据库是有用的。这些架构中,假设所有记录的大小是相同的都是毫无用处的,因为对象可以包含表,可以导致各种大小的对象相同的类。从关系型数据库中选择的数据是在操作的主键和外键的基础上。对象之间的关系可能更复杂。SDDS的所有数据通常存储在主内存中,这是不持久的。为了实现持久的对象上的备份硬盘驱动器是必需的。


在另一边,可扩展性,高运算速度和容错能力是SDDS很重要的优势\cite{7,8,10,11}。一个面向对象的数据库存储在SDDS应该提高OODBMS的性能和使OODB的应用程序可扩展。开发一个保留SDDS所有优势的架构但有效地适用于存储的对象将是非常有用的。


有几个众所周知的SDDS的实现:
\begin{itemize}
\item[-] AMOS-SDDS:可扩展的分布式相结合的制度AMOS II和SDDS-2000,对象 - 关系DMBS与RP *文件作为外部存储\cite{12,13},
\item[-] LH* G的原型运行在一个单一的多线程计算机使用整数作为数据的过程\cite{14},
\item[-] 分布式动态的散列\cite{9},
\item[-] “演员数据库”,演员们操作和存储的SDDS数据在CTH*(分布式小型特里散列)\cite{15}和其他的基础上。
\end{itemize}
然而,在上述提到的论文\cite{9,12,13,14,15}中,RP *架构面向对象的版本没有提及。


\section{RP*对象}
在RP*架构的一个记录应保存的桶地址在桶\cite{7}的范围的基础上计算。因此,该算法计算的记录在SDDS文件(桶的地址)的地址是很简单的。


在OORP*架构中应该存储一个对象的桶的地址也在桶的范围的基础上计算。同一个类的对象可以包含让它们的大小不同的不同尺寸的表。因此,桶应存储对象的各种数字。


因为这个原因,一个分裂的桶应做的:
\begin{enumerate}
\item 一个对象插入
\begin{enumerate}
\item[a)]	桶的负载系数在控制分割的阈值以上;负载系数测量作为桶容量桶中的所有对象的大小的总和的商,
\item[b)] 	没有足够的自由空间用于下一个对象不受控制的分裂,或
\end{enumerate}
\item 一个对象被更新。更新的对象可以改变它的大小。如果更新的对象是更大的话,就可能发生如下:
\begin{enumerate}
\item[a)]	桶的负载系数在控制分割的阈值以上;负载系数测量作为桶容量桶中的所有对象的大小的总和的商,
\item[b)] 	在桶中没有足够的自由空间且在桶分裂过程中使用相同的算法插入的对象。
\end{enumerate}
\end{enumerate}
如果更新的对象是较小或具有相同的大小,它是直接存储桶中。


分割算法用于OORP*(算法1) \cite{16}是类似于原始RP*\cite{7},而且也很简单。但是,它假定$bucketSize ≫ maximum( sizeOf(object))$.


步骤2,3,4和6是在\cite{7}中。在OORP*当一个对象被添加,更新或删除时$bucketSize$必须实现。


\section{Java对象RP*实现}
一个用于Java对象的OORP*体系结构的实施应允许存储,更新,检索和删除单个对象由用户定义的类,限制类上,这可能是容易被接受(也就是类必须是可序列化的)。


OORP*存储的对象分布在多台服务器。因此,他们应该有一些额外的功能。这可以以不同的方式来实现。这些是如下:
\begin{itemize}
\item[-] 修改类的源代码,需要访问这个源代码,
\item[-] 修改Java虚拟机(JVM),但不是所有的许可证允许这样做,另外永久改变JVM会影响其他应用程序,
\item[-] 编译后的字节代码的修改,以获得所需的特性,意味着它的源代码是没有必要的,保持不变。
\end{itemize}


算法1 OORP*分割算法
\begin{enumerate}
\item 在溢满桶B中确定中间键c.
\begin{lstlisting}
size ← 0;
halfBucketSize ← bucketSize/2;
  	while size < halfBucketSize do
  		o ← nextObject(B);
  		size ← size + sizeOf(o);
  		cm ← c(o);
  end while
\end{lstlisting}
\item 尝试创建桶M。如果桶M已经存在,等待确认,或者拒绝。
\item 如果创建被拒绝,则$M←M +1$,请转至步骤2。
\item 复制到桶M 以下内容:
\begin{itemize}
\item[-] 标题:\\
$\lambda \leftarrow c_m(B);$\\
$\wedge \leftarrow \wedge(B);$
\item[-] B的每一个对象用$c > c_m$
\end{itemize}
\item 减少B中最大的键:\\
  $\wedge \leftarrow \wedge(B);$\\
  删除对象移动到桶M,和实现bucketSize。
\item 设置$M←M +1$,\\
  其中:
\begin{itemize}
\item[-] $size$和$halfBucketSize$表示整数变量,
\item[-] $o$表示一个对象变量,
\item[-] $M$表示一个新的桶数,
\item[-] $bucketSize$表示B中的所有对象的大小总和,
\item[-] $c$和$c_m$在B中分别表示一个键和中间键,
\item[-] $\lambda$和$\wedge$分别表示B的最小值和最大值的键,
\item[-] $nextObject$和$sizeof$分别表示B中的下一个对象和对象的大小。
\end{itemize}
\end{enumerate}


为了存储一些类的对象在OORP*类必须包含一些额外的方法和属性。继承基础上的解决方案在这里不适用,因为Java类只能扩展一个基类。程序员可能手动添加这些方法和属性,但它不包括透明度。出于这些原因,最后从上述选项是选择字节代码修改。


有可分辨两种类型的对象,其可以被添加到OORP*:
\begin{itemize}
\item[-] 主对象可以直接添加到OORP*,这些是由用户创建的类的对象,
\item[-] 次要对象可以添加到OORP*仅适用于主要对象,例如Integer类的对象。
\end{itemize}


对象可能会引用到其他主要或次要对象组织成一个列表,例如。管理列表作为一个单独的对象,可能会导致许多与性能相关的问题和依赖的商店。每次当列表中的第一个元素是必需的,将检索整个列表。这是非常耗时的(且效率不高)。为了避免这种问题主对象的引用是用特殊的方式处理。当一个对象添加到OORP*,所有主要的对象是单独添加。当对象被从OORP*检索,没有任何其他主要对象被检索。 当一个包含主要对象访问(读或写)的自动生成方法($getter$或$setter4$)被调用。然后从OORP*中检索对象(存储)。类包含的主要对象修改:
\begin{itemize}
\item[-] 自动生成的表中添加的主要对象标识符,
\item[-] 每一次访问(读或写)这样的字段将被替换为自动生成的方法($getter$或$setter$)的调用,
\item[-] 自动生成$getter$和$setter$的每一个字段包含的主要对象。
\end{itemize}


称为SDDSModifier的程序修改编译的类,并添加所有所需的功能。对象存储在序列化形式。它们被转换成字节的表,发送,然后存储在服务器上的桶。由于它的普及性和普遍性Java集合被选为一个访问对象的方法。这些集合,除了表之外,很可能是第二个主要的方法来安排对象。所有标准的Java集合实现以下两个接口:集合是一个列表和集合的基础,映射是用来映射键到值。接口SDDSCollection,延伸集合,类SDDSFile,实现它开发了。SDDSFile不支持基于对象的字段值的查询,但可能是迭代的对象。


总之,可扩展性的发展,分布式存储的Java对象包括以下三个步骤:
\begin{enumerate}
\item 首先,程序员开发一种使用SDDSFile存储对象的应用程序。应用程序可以使用类存储在OORP*类和其他未与OORP*联系的类。
\item 接下来,编译类使用标准的Java编译器,然后修改SDDSModifier的。
\item 最后,在启动服务器后应用程序可能会启动。每个服务器可能在文字或图形模式下工作。
\end{enumerate}


\section{绩效评估}
在实验对象被插入和检索:
\begin{itemize}
\item[-] 一个的集合(java.util.HashSet),它被存储在主存储器中,
\item[-] 一个文件,该文件被存储在辅助存储器中,
\item[-] RP*N的4个或8个服务器组成的文件。只有一个客户端RP*N文件假定。
\end{itemize}


这些操作的性能进行了测定和比较。在实验过程中,一个类包含的字节表($512 KB$)。存储对象的数目是从$1000$至$6000$(从约$512MB$至约$3GB$)。所有用来测试的类都没有重载方法hashCode()。每个试验在相同的计算机上重复三次:Athlon3.0GHz处理器,1.5 GB RAM和硬盘 - ST380211AS,通过千兆以太网网络连接。实验结果的示图 \ref{fig:subfig1}和\ref{fig:subfig2}。
\begin{figure}[!htbp]
	\centering
	\caption{存储和检索对象时间}
	\subfigure[存储和检索的所有对象的总时间]{
		\label{fig:subfig1:a}
    		\includegraphics[scale=0.4]{figs/pic1.pdf}}
    	\subfigure[存储和检索单个对象的平均时间]{
    		\label{fig:subfig1:b}
    		\includegraphics[scale=0.4]{figs/pic2.pdf}}
    	\label{fig:subfig1}
\end{figure}
\begin{figure}[!htbp]
	\centering
	\caption{存储对象(S)和从(R)OORP* N为4和8的服务器检索的时间}
	\subfigure[存储的所有对象(S)和从(R)OORP* N为4和8的服务器检索的时间]{
		\label{fig:subfig2:a}
    		\includegraphics[scale=0.4]{figs/pic3.pdf}}
    	\subfigure[存储单个对象(S)和从(R)OORP* N为4和8的服务器检索的平均时间]{
    		\label{fig:subfig2:b}
    		\includegraphics[scale=0.4]{figs/pic4.pdf}}
    \label{fig:subfig2}
\end{figure}


OORP*N存储和检索单个对象具有几乎恒定的平均时间。它不直接依赖于对象的数目。的平均时间的最大值和最小值之间的差异是44%左右适用于由4个服务器组成的OORP* N,并包括约23%适用于由8个服务器组成的OORP*N。序列化到一个文件中,这个差异是超过2000%,且存储在主内存中的对象是超过8000%。序列化到一个文件中,从评价的方法是最慢的,如果对象的总大小超出1.5GB(即是在实验中所使用的计算机的主存储器的大小)。此外,在一个文件中的序列化的对象必须按顺序访问。


对于OORP*N检索对象的平均时间几乎是不变的,它是大约17.7毫秒。存储的所有对象的平均时间是变数较多,约19.4毫秒40.6毫秒。这是一个发送对象的服务器的平均时间和与分裂的平均时间的总和。


让我们假设平均时间发送单一的对象始终是相同的,等于不发生分裂时存储单个对象的平均时间。如果s是分裂期间移动的对象的数量与所有对象的数量的比率,然后涉及分裂的平均时间可以计算为s和一些与网络速度相关的恒定值的算术乘积。图\ref{fig:subfig3}给出的分裂期间移动的对象的数量与所有对象的数量的比率。图\ref{fig:subfig4}是理论计算存储单个对象到OORP*N(标有*)的平均时间与实验获得的结果比较。曲线几乎是相同的。
\begin{figure}[!htbp]
	\centering
	\caption{分裂期间移动的对象的数量(Ns)与所有对象的数量(N)的比率}
    	\includegraphics[scale=0.7]{figs/pic5.pdf}
    	\label{fig:subfig3}
\end{figure}
\begin{figure}[!htbp]
	\centering
	\caption{理论计算存储单个对象到OORP*N(标有*)的平均时间与实验获得的结果比较}
    	\includegraphics[scale=0.7]{figs/pic6.pdf}
    \label{fig:subfig4}
\end{figure}

\section{结论}
OORP*可以作为分布式对象存储的一部分。它可以将Java对象序列化形式。OORP*寻址算法是和RP*相同。然而,桶可以存储各种数目的对象。分割算法必须被修改,且不是基于对象的数目而是基于对象的大小。实验证明,存储和检索单个对象的平均时间对于OORP* N比集合和文件更稳定。平均次数的最大值和最小值之间的差异是由4台服务器组成的OORP* N(44%)比由8台服务器(23%)组成的OORP* N几乎小两倍。检索单个对象的平均时间几乎是恒定的,不依赖于对象或服务器的数目。存储的所有对象的时间是发送对象到服务器的时间和时间分裂的总和。分裂的时间取决于分裂期间移动对象的数目,反过来,取决于桶的大小。大桶更好的结果是因为较少的对象被移动。从实验结果证实了理论计算。


  
SDDS的其他架构的面向对象版本将是我们进一步研究的主题。



\bibliographystyle{unsrt}										% 按引用的先后顺序排列,比较次序为作者,年度和标题
\bibliography{mybib}												% 引用文件数据库在bib.bib文件中

\clearpage     
\end{CJK*}
\end{document}


