%===============指定文档的类别=====================
%%article  	排版科技期刊,短报告,程序文档,邀请函等
%%report		排版多章节的长报告,短片的书籍,博士论文等
%%book		排版书籍
%%slides		排版幻灯片.其中使用了较大的sans serif字体,
%			也可以考虑使用FoilTex来得到相同的效果.
\documentclass[10pt,a4paper]{article}
%===============文档包含的宏集=====================
\usepackage[latin1]{inputenc}						
\usepackage{amsmath}
\usepackage{amsfonts}
\usepackage{amssymb}
\usepackage{graphicx}
%===============文档的标题及作者===================
\author{HJY}
\title{custom instruction}
%=================文档的正文======================
\begin{document}
%------------------------
\maketitle							%生成标题
\tableofcontents						%生成目录
%------------------------
\section{charactors}
\subsection{space}
%将空格和制表符等空白字符视为相同的空白距离
I can not stop loving     you.
\newline
%用\\或\newline来表示回车换行,多个空行和一个空行的作用一样.
I can not stop loving you.
%命令\\*				强行断行,还禁止分页
%命令\newpage		另起一新页
%\mbox{text}			保证text的内容在统一行[邮编,电话号码,文件名etc
%句号前的\@			说明这个句号是句子的末尾.
%命令\frenchspacing	能禁止在句号后插入额外的空间
%命令\linespread{factor}		行距{比例}
%命令\setlength{\parindent}{0pt}		设置段落之间的距离为opt
%命令\indent \noindent		段落首行缩进  首行不缩进
%命令\hspace{}				设置水平距离
%命令\vspace{}				设置竖直距离	

\subsection{special charactors}
%特殊字符如下
\# \$ \% \^{} \& \_ \{ \} \~{}


\subsection{quotation}
%单引号(左右一样)'
%双引号(左右一样)''
''Please press the 'x' key.''


\subsection{dash \& hypen}
%符号-		连字号/减号
%符号--		短破折号
%符号---		长破折号
daughter-in-law, X-rated\\
pages 13--67\\
yes---or no? \\
$0$, $1$ and $-1$


\subsection{wave}
%符号\~			波浪号
%命令\sim		波浪号(数学符号)
http://www.rich.edu/\~{}bush \\
http://www.clever.edu/$\sim$demo


\subsection{degree}
%命令$\,^{circ}\mathrm{C}$	摄氏度
Its $-30\,^{\circ}\mathrm{C}$,I will soon start to super-conduct.


\subsection{ellipsis}
%命令$\ldots 	省略号
Not like this ... but like this:\\
New York, Tokyo, Budapest, \ldots


\subsection{reference}
%\label{marker}    	标识符
%\ref{marker}		引用
%\pageref{marker}	排印\laber输入处的页码
A reference to this subsection
\label{sec:this} looks like:
''see section~\ref{sec:this} on
page~\pageref{sec:this}.''


\subsection{footnote}
%\footnote{text}			位于当前页的页脚位置的脚注	
Footnotes\footnote{This is
a footnote.} are often used
by people using \LaTeX.


\subsection{emphasize}
%命令\underline{text}		用下划线来强调重要的单词
%命令\emph{text}			用斜体来强调重要单词	
\emph{If you use
emphasizing inside a piece
of emphasized text, then
\LaTeX{} uses the
\underline{normal} font for
emphasizing.}



\section{Environment}
\subsection{list}
%itemize					用于简单的列表
%enumerate				用于带序号的列表
%description				用于带描述的列表
\begin{enumerate}
\item You can mix the list
environments to your taste:
\begin{itemize}
\item But it might start to
look silly.
\item[-] With a dash.
\end{itemize}
\item Therefore remember:
\begin{description}
\item[Stupid] things will not
become smart because they are
in a list.
\item[Smart] things, though, can be
presented beautifully in a list.
\end{description}
\end{enumerate}


\subsection{alignment}
%flushleft					左对齐
%flushright					右对齐
%center						居中
\begin{flushright}
This text is right-\\aligned.
\LaTeX{} is not trying to make
each line the same length.
\end{flushright}


\subsection{quotation}
%quote						用于重要的断句和例子的引用
%quotation					用于超过几段的较长的引用(它对段落进行缩进)
%verse						用于诗歌,在诗歌中断行很重要
A typographical rule of thumb
for the line length is:
\begin{quote}
On average, no line should
be longer than 66 characters.
\end{quote}


\subsection{verbatim}
%verbatim					将文本直接打印,包括所有的断行和空白
%\verb|text|					将text原文显示
\verb|like   this :-) |


\subsection{table}
%tabular						用来排印带有水平和铅直表线的漂亮表格
%	l r c					左 右 中对齐
%	p{value}					相应的宽度
%	|						产生铅直表线
%	&						跳入下一列
% 	\\						开始新的一行
%	hline					插入水平表线
%	\cline{j-i}				表示表线的起始列和终止列的序号
%		h					放置说明符-放置在当前页
%		t					放置说明符-放置在页面顶部
%		b					放置说明符-放置在页面底部
%		!					放置说明符-忽略阻止浮动体放置的大部分内部参数
\begin{tabular}{|r|l|}
\hline
7C0 & hexadecimal \\
3700 & octal \\ \cline{2-2}
11111000000 & binary \\
\hline \hline
1984 & decimal \\
\hline
\end{tabular}

\begin{tabular}{c r @{.} l} 
Pi expression 
& 
\multicolumn{2}{c}{Value} \\ 
\hline 
$\pi$ 
& 3&1416 \\ 
$\pi^{\pi}$ 
& 36&46 
\\ 
$(\pi^{\pi})^{\pi}$ & 80662&7 \\ 
\end{tabular} 


\section{mathematics}
\subsection{base knowledge}
%$express$					数学表达式
\TeX{} is pronounced as
$\tau\epsilon\chi$.\\[6pt]
100~m$^{3}$ of water\\[6pt]
This comes from my $\heartsuit$

\ldots when Einstein introduced his formula
%没有编号的显示式样
\begin{displaymath}
c^{2}=a^{2}+b^{2}
\end{displaymath}


\subsection{function}
%\sqrt 						平方根
%\overline \underline		上下划线
%\overbrace \underlince		上下方水平的大括号
%\vec  \overrightarrow \overleftarrow 向量
%\frac						分数
%\int \sum					积分 求和
%有编号的显示式样
\begin{eqnarray}
e = m \cdot c^2 \; \\
\sum_{k=1}^{n} I_k = 0 \; \\
\forall x \in \mathbf{R}:
\end{eqnarray}
\begin{equation}
e = m \cdot c^2 \; \\
\sum_{k=1}^{n} I_k = 0 \; \\
\end{equation}


\subsection{Lowercase Greek letters}
$\lambda,\xi,\pi,\mu,\Phi,\Omega$


\subsection*{theorem}
%\newtheorem{name}[counter]{text}				定理name-唯一标识,counter-个数,text-显示的定理名
\newtheorem{law}{Law}
\newtheorem{jury}[law]{Jury}
\begin{law} \label{law:box}
Don't hide in the witness box
\end{law}
\begin{jury}[The Twelve]
It could be you! So beware and
see law~\ref{law:box}\end{jury}
\begin{law}No, No, No\end{law}



\section{graphy}
%在文档中添加图片
\begin{figure}[h]
\begin{center}
\includegraphics[angle=90, width=0.5\textwidth]{test}
\end{center}
\end{figure}


\section{install package}
\begin{enumerate}
\item	download package from $CTAN(http://www.ctan.org/)$
\item	install	--tar files into /usr/share/texmf/tex/latex
\item   update	--tar files into /usr/share/texmf-texlive/tex/latex(maybe you should backup the original files)
\item	mark the packages--perform the instruction \underline{texhash} in the terminal.
\item	log off /log on
\end{enumerate}


\section{My Latex}
\subsection{my command}
\newcommand{\txsit}[1]
{This is the \emph{#1} Short
Introduction to \LaTeXe}
% in the document body:
\begin{itemize}
\item \txsit{not so}
\item \txsit{very}
\end{itemize}


\subsection{my environment}
\newenvironment{king}
{\rule{1ex}{1ex}%
\hspace{\stretch{1}}}
{\hspace{\stretch{1}}%
\rule{1ex}{1ex}}
\begin{king}
My humble subjects \ldots
\end{king}


\section{bibliographystyle}
%plain 	按字母的顺序排列,比较次序为作者、年度和标题.
%unsrt	样式同plain,只是按照引用的先后排序.
%alpha	用作者名首字母+年份后两位作标号,以字母顺序排序.
%abbrv	类似plain,将月份全拼改为缩写,更显紧凑.
%ieeetr	国际电气电子工程师协会期刊样式.
%acm		美国计算机学会期刊样式.
%siam	美国工业和应用数学学会期刊样式.
%apalike	美国心理学学会期刊样式.
shownothing
\end{document}
